
% Default to the notebook output style

    


% Inherit from the specified cell style.




    
\documentclass[11pt]{article}

    
    
    \usepackage[T1]{fontenc}
    % Nicer default font (+ math font) than Computer Modern for most use cases
    \usepackage{mathpazo}

    % Basic figure setup, for now with no caption control since it's done
    % automatically by Pandoc (which extracts ![](path) syntax from Markdown).
    \usepackage{graphicx}
    % We will generate all images so they have a width \maxwidth. This means
    % that they will get their normal width if they fit onto the page, but
    % are scaled down if they would overflow the margins.
    \makeatletter
    \def\maxwidth{\ifdim\Gin@nat@width>\linewidth\linewidth
    \else\Gin@nat@width\fi}
    \makeatother
    \let\Oldincludegraphics\includegraphics
    % Set max figure width to be 80% of text width, for now hardcoded.
    \renewcommand{\includegraphics}[1]{\Oldincludegraphics[width=.8\maxwidth]{#1}}
    % Ensure that by default, figures have no caption (until we provide a
    % proper Figure object with a Caption API and a way to capture that
    % in the conversion process - todo).
    \usepackage{caption}
    \DeclareCaptionLabelFormat{nolabel}{}
    \captionsetup{labelformat=nolabel}

    \usepackage{adjustbox} % Used to constrain images to a maximum size 
    \usepackage{xcolor} % Allow colors to be defined
    \usepackage{enumerate} % Needed for markdown enumerations to work
    \usepackage{geometry} % Used to adjust the document margins
    \usepackage{amsmath} % Equations
    \usepackage{amssymb} % Equations
    \usepackage{textcomp} % defines textquotesingle
    % Hack from http://tex.stackexchange.com/a/47451/13684:
    \AtBeginDocument{%
        \def\PYZsq{\textquotesingle}% Upright quotes in Pygmentized code
    }
    \usepackage{upquote} % Upright quotes for verbatim code
    \usepackage{eurosym} % defines \euro
    \usepackage[mathletters]{ucs} % Extended unicode (utf-8) support
    \usepackage[utf8x]{inputenc} % Allow utf-8 characters in the tex document
    \usepackage{fancyvrb} % verbatim replacement that allows latex
    \usepackage{grffile} % extends the file name processing of package graphics 
                         % to support a larger range 
    % The hyperref package gives us a pdf with properly built
    % internal navigation ('pdf bookmarks' for the table of contents,
    % internal cross-reference links, web links for URLs, etc.)
    \usepackage{hyperref}
    \usepackage{longtable} % longtable support required by pandoc >1.10
    \usepackage{booktabs}  % table support for pandoc > 1.12.2
    \usepackage[inline]{enumitem} % IRkernel/repr support (it uses the enumerate* environment)
    \usepackage[normalem]{ulem} % ulem is needed to support strikethroughs (\sout)
                                % normalem makes italics be italics, not underlines
    

    
    
    % Colors for the hyperref package
    \definecolor{urlcolor}{rgb}{0,.145,.698}
    \definecolor{linkcolor}{rgb}{.71,0.21,0.01}
    \definecolor{citecolor}{rgb}{.12,.54,.11}

    % ANSI colors
    \definecolor{ansi-black}{HTML}{3E424D}
    \definecolor{ansi-black-intense}{HTML}{282C36}
    \definecolor{ansi-red}{HTML}{E75C58}
    \definecolor{ansi-red-intense}{HTML}{B22B31}
    \definecolor{ansi-green}{HTML}{00A250}
    \definecolor{ansi-green-intense}{HTML}{007427}
    \definecolor{ansi-yellow}{HTML}{DDB62B}
    \definecolor{ansi-yellow-intense}{HTML}{B27D12}
    \definecolor{ansi-blue}{HTML}{208FFB}
    \definecolor{ansi-blue-intense}{HTML}{0065CA}
    \definecolor{ansi-magenta}{HTML}{D160C4}
    \definecolor{ansi-magenta-intense}{HTML}{A03196}
    \definecolor{ansi-cyan}{HTML}{60C6C8}
    \definecolor{ansi-cyan-intense}{HTML}{258F8F}
    \definecolor{ansi-white}{HTML}{C5C1B4}
    \definecolor{ansi-white-intense}{HTML}{A1A6B2}

    % commands and environments needed by pandoc snippets
    % extracted from the output of `pandoc -s`
    \providecommand{\tightlist}{%
      \setlength{\itemsep}{0pt}\setlength{\parskip}{0pt}}
    \DefineVerbatimEnvironment{Highlighting}{Verbatim}{commandchars=\\\{\}}
    % Add ',fontsize=\small' for more characters per line
    \newenvironment{Shaded}{}{}
    \newcommand{\KeywordTok}[1]{\textcolor[rgb]{0.00,0.44,0.13}{\textbf{{#1}}}}
    \newcommand{\DataTypeTok}[1]{\textcolor[rgb]{0.56,0.13,0.00}{{#1}}}
    \newcommand{\DecValTok}[1]{\textcolor[rgb]{0.25,0.63,0.44}{{#1}}}
    \newcommand{\BaseNTok}[1]{\textcolor[rgb]{0.25,0.63,0.44}{{#1}}}
    \newcommand{\FloatTok}[1]{\textcolor[rgb]{0.25,0.63,0.44}{{#1}}}
    \newcommand{\CharTok}[1]{\textcolor[rgb]{0.25,0.44,0.63}{{#1}}}
    \newcommand{\StringTok}[1]{\textcolor[rgb]{0.25,0.44,0.63}{{#1}}}
    \newcommand{\CommentTok}[1]{\textcolor[rgb]{0.38,0.63,0.69}{\textit{{#1}}}}
    \newcommand{\OtherTok}[1]{\textcolor[rgb]{0.00,0.44,0.13}{{#1}}}
    \newcommand{\AlertTok}[1]{\textcolor[rgb]{1.00,0.00,0.00}{\textbf{{#1}}}}
    \newcommand{\FunctionTok}[1]{\textcolor[rgb]{0.02,0.16,0.49}{{#1}}}
    \newcommand{\RegionMarkerTok}[1]{{#1}}
    \newcommand{\ErrorTok}[1]{\textcolor[rgb]{1.00,0.00,0.00}{\textbf{{#1}}}}
    \newcommand{\NormalTok}[1]{{#1}}
    
    % Additional commands for more recent versions of Pandoc
    \newcommand{\ConstantTok}[1]{\textcolor[rgb]{0.53,0.00,0.00}{{#1}}}
    \newcommand{\SpecialCharTok}[1]{\textcolor[rgb]{0.25,0.44,0.63}{{#1}}}
    \newcommand{\VerbatimStringTok}[1]{\textcolor[rgb]{0.25,0.44,0.63}{{#1}}}
    \newcommand{\SpecialStringTok}[1]{\textcolor[rgb]{0.73,0.40,0.53}{{#1}}}
    \newcommand{\ImportTok}[1]{{#1}}
    \newcommand{\DocumentationTok}[1]{\textcolor[rgb]{0.73,0.13,0.13}{\textit{{#1}}}}
    \newcommand{\AnnotationTok}[1]{\textcolor[rgb]{0.38,0.63,0.69}{\textbf{\textit{{#1}}}}}
    \newcommand{\CommentVarTok}[1]{\textcolor[rgb]{0.38,0.63,0.69}{\textbf{\textit{{#1}}}}}
    \newcommand{\VariableTok}[1]{\textcolor[rgb]{0.10,0.09,0.49}{{#1}}}
    \newcommand{\ControlFlowTok}[1]{\textcolor[rgb]{0.00,0.44,0.13}{\textbf{{#1}}}}
    \newcommand{\OperatorTok}[1]{\textcolor[rgb]{0.40,0.40,0.40}{{#1}}}
    \newcommand{\BuiltInTok}[1]{{#1}}
    \newcommand{\ExtensionTok}[1]{{#1}}
    \newcommand{\PreprocessorTok}[1]{\textcolor[rgb]{0.74,0.48,0.00}{{#1}}}
    \newcommand{\AttributeTok}[1]{\textcolor[rgb]{0.49,0.56,0.16}{{#1}}}
    \newcommand{\InformationTok}[1]{\textcolor[rgb]{0.38,0.63,0.69}{\textbf{\textit{{#1}}}}}
    \newcommand{\WarningTok}[1]{\textcolor[rgb]{0.38,0.63,0.69}{\textbf{\textit{{#1}}}}}
    
    
    % Define a nice break command that doesn't care if a line doesn't already
    % exist.
    \def\br{\hspace*{\fill} \\* }
    % Math Jax compatability definitions
    \def\gt{>}
    \def\lt{<}
    % Document parameters
    \title{Analise\_Bay\_Area\_Bike\_Share-NDFDSI}
    
    
    

    % Pygments definitions
    
\makeatletter
\def\PY@reset{\let\PY@it=\relax \let\PY@bf=\relax%
    \let\PY@ul=\relax \let\PY@tc=\relax%
    \let\PY@bc=\relax \let\PY@ff=\relax}
\def\PY@tok#1{\csname PY@tok@#1\endcsname}
\def\PY@toks#1+{\ifx\relax#1\empty\else%
    \PY@tok{#1}\expandafter\PY@toks\fi}
\def\PY@do#1{\PY@bc{\PY@tc{\PY@ul{%
    \PY@it{\PY@bf{\PY@ff{#1}}}}}}}
\def\PY#1#2{\PY@reset\PY@toks#1+\relax+\PY@do{#2}}

\expandafter\def\csname PY@tok@gd\endcsname{\def\PY@tc##1{\textcolor[rgb]{0.63,0.00,0.00}{##1}}}
\expandafter\def\csname PY@tok@gu\endcsname{\let\PY@bf=\textbf\def\PY@tc##1{\textcolor[rgb]{0.50,0.00,0.50}{##1}}}
\expandafter\def\csname PY@tok@gt\endcsname{\def\PY@tc##1{\textcolor[rgb]{0.00,0.27,0.87}{##1}}}
\expandafter\def\csname PY@tok@gs\endcsname{\let\PY@bf=\textbf}
\expandafter\def\csname PY@tok@gr\endcsname{\def\PY@tc##1{\textcolor[rgb]{1.00,0.00,0.00}{##1}}}
\expandafter\def\csname PY@tok@cm\endcsname{\let\PY@it=\textit\def\PY@tc##1{\textcolor[rgb]{0.25,0.50,0.50}{##1}}}
\expandafter\def\csname PY@tok@vg\endcsname{\def\PY@tc##1{\textcolor[rgb]{0.10,0.09,0.49}{##1}}}
\expandafter\def\csname PY@tok@vi\endcsname{\def\PY@tc##1{\textcolor[rgb]{0.10,0.09,0.49}{##1}}}
\expandafter\def\csname PY@tok@vm\endcsname{\def\PY@tc##1{\textcolor[rgb]{0.10,0.09,0.49}{##1}}}
\expandafter\def\csname PY@tok@mh\endcsname{\def\PY@tc##1{\textcolor[rgb]{0.40,0.40,0.40}{##1}}}
\expandafter\def\csname PY@tok@cs\endcsname{\let\PY@it=\textit\def\PY@tc##1{\textcolor[rgb]{0.25,0.50,0.50}{##1}}}
\expandafter\def\csname PY@tok@ge\endcsname{\let\PY@it=\textit}
\expandafter\def\csname PY@tok@vc\endcsname{\def\PY@tc##1{\textcolor[rgb]{0.10,0.09,0.49}{##1}}}
\expandafter\def\csname PY@tok@il\endcsname{\def\PY@tc##1{\textcolor[rgb]{0.40,0.40,0.40}{##1}}}
\expandafter\def\csname PY@tok@go\endcsname{\def\PY@tc##1{\textcolor[rgb]{0.53,0.53,0.53}{##1}}}
\expandafter\def\csname PY@tok@cp\endcsname{\def\PY@tc##1{\textcolor[rgb]{0.74,0.48,0.00}{##1}}}
\expandafter\def\csname PY@tok@gi\endcsname{\def\PY@tc##1{\textcolor[rgb]{0.00,0.63,0.00}{##1}}}
\expandafter\def\csname PY@tok@gh\endcsname{\let\PY@bf=\textbf\def\PY@tc##1{\textcolor[rgb]{0.00,0.00,0.50}{##1}}}
\expandafter\def\csname PY@tok@ni\endcsname{\let\PY@bf=\textbf\def\PY@tc##1{\textcolor[rgb]{0.60,0.60,0.60}{##1}}}
\expandafter\def\csname PY@tok@nl\endcsname{\def\PY@tc##1{\textcolor[rgb]{0.63,0.63,0.00}{##1}}}
\expandafter\def\csname PY@tok@nn\endcsname{\let\PY@bf=\textbf\def\PY@tc##1{\textcolor[rgb]{0.00,0.00,1.00}{##1}}}
\expandafter\def\csname PY@tok@no\endcsname{\def\PY@tc##1{\textcolor[rgb]{0.53,0.00,0.00}{##1}}}
\expandafter\def\csname PY@tok@na\endcsname{\def\PY@tc##1{\textcolor[rgb]{0.49,0.56,0.16}{##1}}}
\expandafter\def\csname PY@tok@nb\endcsname{\def\PY@tc##1{\textcolor[rgb]{0.00,0.50,0.00}{##1}}}
\expandafter\def\csname PY@tok@nc\endcsname{\let\PY@bf=\textbf\def\PY@tc##1{\textcolor[rgb]{0.00,0.00,1.00}{##1}}}
\expandafter\def\csname PY@tok@nd\endcsname{\def\PY@tc##1{\textcolor[rgb]{0.67,0.13,1.00}{##1}}}
\expandafter\def\csname PY@tok@ne\endcsname{\let\PY@bf=\textbf\def\PY@tc##1{\textcolor[rgb]{0.82,0.25,0.23}{##1}}}
\expandafter\def\csname PY@tok@nf\endcsname{\def\PY@tc##1{\textcolor[rgb]{0.00,0.00,1.00}{##1}}}
\expandafter\def\csname PY@tok@si\endcsname{\let\PY@bf=\textbf\def\PY@tc##1{\textcolor[rgb]{0.73,0.40,0.53}{##1}}}
\expandafter\def\csname PY@tok@s2\endcsname{\def\PY@tc##1{\textcolor[rgb]{0.73,0.13,0.13}{##1}}}
\expandafter\def\csname PY@tok@nt\endcsname{\let\PY@bf=\textbf\def\PY@tc##1{\textcolor[rgb]{0.00,0.50,0.00}{##1}}}
\expandafter\def\csname PY@tok@nv\endcsname{\def\PY@tc##1{\textcolor[rgb]{0.10,0.09,0.49}{##1}}}
\expandafter\def\csname PY@tok@s1\endcsname{\def\PY@tc##1{\textcolor[rgb]{0.73,0.13,0.13}{##1}}}
\expandafter\def\csname PY@tok@dl\endcsname{\def\PY@tc##1{\textcolor[rgb]{0.73,0.13,0.13}{##1}}}
\expandafter\def\csname PY@tok@ch\endcsname{\let\PY@it=\textit\def\PY@tc##1{\textcolor[rgb]{0.25,0.50,0.50}{##1}}}
\expandafter\def\csname PY@tok@m\endcsname{\def\PY@tc##1{\textcolor[rgb]{0.40,0.40,0.40}{##1}}}
\expandafter\def\csname PY@tok@gp\endcsname{\let\PY@bf=\textbf\def\PY@tc##1{\textcolor[rgb]{0.00,0.00,0.50}{##1}}}
\expandafter\def\csname PY@tok@sh\endcsname{\def\PY@tc##1{\textcolor[rgb]{0.73,0.13,0.13}{##1}}}
\expandafter\def\csname PY@tok@ow\endcsname{\let\PY@bf=\textbf\def\PY@tc##1{\textcolor[rgb]{0.67,0.13,1.00}{##1}}}
\expandafter\def\csname PY@tok@sx\endcsname{\def\PY@tc##1{\textcolor[rgb]{0.00,0.50,0.00}{##1}}}
\expandafter\def\csname PY@tok@bp\endcsname{\def\PY@tc##1{\textcolor[rgb]{0.00,0.50,0.00}{##1}}}
\expandafter\def\csname PY@tok@c1\endcsname{\let\PY@it=\textit\def\PY@tc##1{\textcolor[rgb]{0.25,0.50,0.50}{##1}}}
\expandafter\def\csname PY@tok@fm\endcsname{\def\PY@tc##1{\textcolor[rgb]{0.00,0.00,1.00}{##1}}}
\expandafter\def\csname PY@tok@o\endcsname{\def\PY@tc##1{\textcolor[rgb]{0.40,0.40,0.40}{##1}}}
\expandafter\def\csname PY@tok@kc\endcsname{\let\PY@bf=\textbf\def\PY@tc##1{\textcolor[rgb]{0.00,0.50,0.00}{##1}}}
\expandafter\def\csname PY@tok@c\endcsname{\let\PY@it=\textit\def\PY@tc##1{\textcolor[rgb]{0.25,0.50,0.50}{##1}}}
\expandafter\def\csname PY@tok@mf\endcsname{\def\PY@tc##1{\textcolor[rgb]{0.40,0.40,0.40}{##1}}}
\expandafter\def\csname PY@tok@err\endcsname{\def\PY@bc##1{\setlength{\fboxsep}{0pt}\fcolorbox[rgb]{1.00,0.00,0.00}{1,1,1}{\strut ##1}}}
\expandafter\def\csname PY@tok@mb\endcsname{\def\PY@tc##1{\textcolor[rgb]{0.40,0.40,0.40}{##1}}}
\expandafter\def\csname PY@tok@ss\endcsname{\def\PY@tc##1{\textcolor[rgb]{0.10,0.09,0.49}{##1}}}
\expandafter\def\csname PY@tok@sr\endcsname{\def\PY@tc##1{\textcolor[rgb]{0.73,0.40,0.53}{##1}}}
\expandafter\def\csname PY@tok@mo\endcsname{\def\PY@tc##1{\textcolor[rgb]{0.40,0.40,0.40}{##1}}}
\expandafter\def\csname PY@tok@kd\endcsname{\let\PY@bf=\textbf\def\PY@tc##1{\textcolor[rgb]{0.00,0.50,0.00}{##1}}}
\expandafter\def\csname PY@tok@mi\endcsname{\def\PY@tc##1{\textcolor[rgb]{0.40,0.40,0.40}{##1}}}
\expandafter\def\csname PY@tok@kn\endcsname{\let\PY@bf=\textbf\def\PY@tc##1{\textcolor[rgb]{0.00,0.50,0.00}{##1}}}
\expandafter\def\csname PY@tok@cpf\endcsname{\let\PY@it=\textit\def\PY@tc##1{\textcolor[rgb]{0.25,0.50,0.50}{##1}}}
\expandafter\def\csname PY@tok@kr\endcsname{\let\PY@bf=\textbf\def\PY@tc##1{\textcolor[rgb]{0.00,0.50,0.00}{##1}}}
\expandafter\def\csname PY@tok@s\endcsname{\def\PY@tc##1{\textcolor[rgb]{0.73,0.13,0.13}{##1}}}
\expandafter\def\csname PY@tok@kp\endcsname{\def\PY@tc##1{\textcolor[rgb]{0.00,0.50,0.00}{##1}}}
\expandafter\def\csname PY@tok@w\endcsname{\def\PY@tc##1{\textcolor[rgb]{0.73,0.73,0.73}{##1}}}
\expandafter\def\csname PY@tok@kt\endcsname{\def\PY@tc##1{\textcolor[rgb]{0.69,0.00,0.25}{##1}}}
\expandafter\def\csname PY@tok@sc\endcsname{\def\PY@tc##1{\textcolor[rgb]{0.73,0.13,0.13}{##1}}}
\expandafter\def\csname PY@tok@sb\endcsname{\def\PY@tc##1{\textcolor[rgb]{0.73,0.13,0.13}{##1}}}
\expandafter\def\csname PY@tok@sa\endcsname{\def\PY@tc##1{\textcolor[rgb]{0.73,0.13,0.13}{##1}}}
\expandafter\def\csname PY@tok@k\endcsname{\let\PY@bf=\textbf\def\PY@tc##1{\textcolor[rgb]{0.00,0.50,0.00}{##1}}}
\expandafter\def\csname PY@tok@se\endcsname{\let\PY@bf=\textbf\def\PY@tc##1{\textcolor[rgb]{0.73,0.40,0.13}{##1}}}
\expandafter\def\csname PY@tok@sd\endcsname{\let\PY@it=\textit\def\PY@tc##1{\textcolor[rgb]{0.73,0.13,0.13}{##1}}}

\def\PYZbs{\char`\\}
\def\PYZus{\char`\_}
\def\PYZob{\char`\{}
\def\PYZcb{\char`\}}
\def\PYZca{\char`\^}
\def\PYZam{\char`\&}
\def\PYZlt{\char`\<}
\def\PYZgt{\char`\>}
\def\PYZsh{\char`\#}
\def\PYZpc{\char`\%}
\def\PYZdl{\char`\$}
\def\PYZhy{\char`\-}
\def\PYZsq{\char`\'}
\def\PYZdq{\char`\"}
\def\PYZti{\char`\~}
% for compatibility with earlier versions
\def\PYZat{@}
\def\PYZlb{[}
\def\PYZrb{]}
\makeatother


    % Exact colors from NB
    \definecolor{incolor}{rgb}{0.0, 0.0, 0.5}
    \definecolor{outcolor}{rgb}{0.545, 0.0, 0.0}



    
    % Prevent overflowing lines due to hard-to-break entities
    \sloppy 
    % Setup hyperref package
    \hypersetup{
      breaklinks=true,  % so long urls are correctly broken across lines
      colorlinks=true,
      urlcolor=urlcolor,
      linkcolor=linkcolor,
      citecolor=citecolor,
      }
    % Slightly bigger margins than the latex defaults
    
    \geometry{verbose,tmargin=1in,bmargin=1in,lmargin=1in,rmargin=1in}
    
    

    \begin{document}
    
    
    \maketitle
    
    

    
    \section{Análise da Bay Area Bike
Share}\label{anuxe1lise-da-bay-area-bike-share}

\subsection{Introdução}\label{introduuxe7uxe3o}

\begin{quote}
\textbf{Dica}: Seções citadas como esta fornecerão instruções úteis
sobre como navegar e usar um notebook do iPython.
\end{quote}

\href{http://www.bayareabikeshare.com/}{Bay Area Bike Share} é uma
empresa que oferece aluguel de bicicletas on-demand para clientes em San
Francisco, Redwood City, Palo Alto, Mountain View e San Jose. Os
usuários podem desbloquear bicicletas de uma variedade de estações em
cada cidade, e devolvê-las em qualquer estação dentro da mesma cidade.
Os usuários pagam o serviço por meio de assinatura anual ou pela compra
de passes de 3 dias ou 24 horas. Os usuários podem fazer um número
ilimitado de viagens. Viagens com menos de trinta minutos de duração não
têm custo adicional; Viagens mais longas incorrem em taxas de horas
extras.

Neste projeto, você vai se colocar no lugar de um analista de dados
realizando uma análise exploratória sobre os dados. Você vai dar uma
olhada em duas das principais partes do processo de análise de dados:
limpeza de dados e análise exploratória. Mas antes que você comece a
olhar os dados, pense algumas perguntas que você pode querer fazer sobre
os dados. Considere, por exemplo, se você estivesse trabalhando para Bay
Area Bike Share: que tipo de informação você gostaria de saber a fim de
tomar decisões de negócios mais inteligentes? Ou você pode pensar se
você fosse um usuário do serviço de compartilhamento de bicicletas. Que
fatores podem influenciar a maneira como você gostaria de usar o
serviço?

    \section{Sobre este projeto}\label{sobre-este-projeto}

Este é o seu primeiro projeto com a Udacity. Queremos fazer com que você
treine os conhecimentos aprendidos durante o curso e que entenda algumas
das dificuldades que pode ter quando for aplicar os mesmos.

Os principais pontos que serão verificados neste trabalho:

\begin{itemize}
\tightlist
\item
  Criação de dicionários e mapeamento de variáveis
\item
  Uso de lógica com o \texttt{if}
\item
  Manipulação de dados e criação de gráficos simples com o
  \texttt{Pandas}
\end{itemize}

\emph{Como conseguir ajuda}: Sugerimos que tente os seguintes canais,
nas seguintes ordens:

\begin{longtable}[]{@{}lllll@{}}
\toprule
Tipo de dúvida\Canais & Google & Fórum & Slack & Email\tabularnewline
\midrule
\endhead
Programação Pyhon e Pandas & 1 & 2 & 3 &\tabularnewline
Requisitos do projeto & & 1 & 2 & 3\tabularnewline
Partes específicas do Projeto & & 1 & 2 & 3\tabularnewline
\bottomrule
\end{longtable}

Os endereços dos canais são:

\begin{itemize}
\tightlist
\item
  Fórum: https://discussions.udacity.com/c/ndfdsi-project
\item
  Slack:
  \href{https://udacity-br.slack.com/messages/C5MT6E3E1}{udacity-br.slack.com}
\item
  Email: data-suporte@udacity.com
\end{itemize}

\textbf{Espera-se que o estudante entregue este relatório com:}

\begin{itemize}
\tightlist
\item
  Todos os TODO feitos, pois eles são essenciais para que o código rode
  corretamente
\item
  Todas as perguntas respondidas. Elas estão identificadas como PERGUNTA
  em letras grandes.
\end{itemize}

Para entregar este projeto, vá a
\href{https://classroom.udacity.com/nanodegrees/nd110/parts/05e3b1e2-ff23-488f-aaec-caba12ad8ed3/modules/519425b3-ee26-4ecd-a952-f689decef51c/lessons/47133342-a203-4df9-9b9e-69b04408e089/project}{sala
de aula} e submeta o seu \texttt{.ipynb} e um pdf, zipados.

    \section{Pergunta 1}\label{pergunta-1}

Escreva pelo menos duas perguntas que você acha que poderiam ser
respondidas usando os dados.

    \textbf{Resposta}: Após a análise de quais informações são fornecidas
nos \href{http://www.bayareabikeshare.com/open-data}{dados}, consegui
pensar nas seguintes perguntas:

\textbf{Questão 1:} Qual é a estação com o menor numero médio de vagas?
Se existir momentos onde não há vagas, é possível determinar os horários
onde essa lotação ocorre? A causa dessa lotação é devido a não locação
das bicicletas da estação ou por muitas devoluções nessa estação? Os
clientes não devem encontrar uma estação sem diponibilidade de vaga. Se
isso ocorrer, não há dados sobre quantos estão ficando sem vagas, visto
que não há como saber o trajeto realizado (apenas ponto de partido e de
devolução) através dos dados fornecidos.

\textbf{Questão 2:} Quais estações recebem mais devoluções de bicicletas
do que retiradas? Por outro lado, quais estações recebem mais retiradas
do que devoluções? Quantos utilizam o seviço para se deslocar entre as
estações e quantos realizam a devolução na mesma estações onde retirou a
bicicleta? Existindo esses défictis, é possível fazer um planejamento e
controlar essee problemas.

\textbf{Questão 3:} Qual o tempo médio de utilização? A maioria dos
usuários utlizam o serviço para um deslocamento curto ou longo? Com qual
frequencia? Essa informação poderia ser comparada com a porcentagem de
clientes casual e anual. Dependendo do resultado, é possível identificar
potenciais clientes anuais que atualmente são do tipo casual.

\textbf{Questão 4:} Qual o número médio de locações realizadas em dias
de chuva compardo a média dos dias sem chuva? Qaul a redução em
porcentagem para dias chuvosos? Qual a porcentagem de dias chuvosos?
Analisando essas duas informações, é possivél prever quanto está sendo
perdido, em média, por conta dos dias chuvosos e assim planejar uma
solução caso esse parâmetro mostre-se significancia.

\begin{quote}
\textbf{Dica}: Se você clicar duas vezes nesta célula, você verá o texto
se alterar removendo toda a formatação. Isso permite editar este bloco
de texto. Este bloco de texto é escrito usando
\href{http://daringfireball.net/projects/markdown/syntax}{Markdown}, que
é uma forma de formatar texto usando cabeçalhos, links, itálico e muitas
outras opções. Pressione \textbf{Shift} + \textbf{Enter} ou
\textbf{Shift} + \textbf{Retorno} para voltar a mostrar o texto
formatado.
\end{quote}

    \subsection{Usando visualizações para comunicar resultados em
dados}\label{usando-visualizauxe7uxf5es-para-comunicar-resultados-em-dados}

Como um analista de dados, a capacidade de comunicar eficazmente
resultados é uma parte fundamental do trabalho. Afinal, sua melhor
análise é tão boa quanto sua capacidade de comunicá-la.

Em 2014, Bay Area Bike Share realizou um
\href{http://www.bayareabikeshare.com/datachallenge-2014}{Open Data
Challenge} para incentivar os analistas de dados a criar visualizações
com base em seu conjunto de dados aberto. Você criará suas próprias
visualizações neste projeto, mas primeiro, dê uma olhada no
\href{http://thfield.github.io/babs/index.html}{vencedor de inscrições
para Melhor Análise} de Tyler Field. Leia todo o relatório para
responder à seguinte pergunta:

    \section{Pergunta 2}\label{pergunta-2}

Que visualizações você acha que fornecem as idéias mais interessantes?
Você é capaz de responder a uma das perguntas identificadas acima com
base na análise de Tyler? Por que ou por que não?

Crie duas visualizações que forneçam idéias interessantes e que
respondam alguma das perguntas da análise de Tyler.

    ** Resposta **:

As visualizações que fornecem as ideia mais interessantes são:

\begin{itemize}
\item
  Número de trajetos, separados em tipos de usuários, para os dias da
  semana e para as horas do dia. (Rides by Weekday e Rides by Hour)
\item
  Número total de trajeto ao longo de todo o período dos dados,
  separados por tipo de usuário e com possibilidade de destacar período
  de tempos específicos como fins de semana, dias de semana, jogos
  esportivos importantes, condições climáticas, acontecimentos
  políticos, greves e feriados nacionais.
\end{itemize}

Considerando as perguntas "identificadas a cima" como sendo as perguntas
formuladas por mim, sim, é possível responder a \textbf{Questão 2},
\textbf{Questão 3}, \textbf{Questão 4}, pois a base dos dados
necessários para essas análises foram plotados pela análise de Taylor.

    \textbf{Visualização 1:} When is Bay Area Bike Share used?

    \begin{Verbatim}[commandchars=\\\{\}]
{\color{incolor}In [{\color{incolor}1}]:} \PY{k+kn}{from} \PY{n+nn}{IPython.display} \PY{k+kn}{import} \PY{n}{Image}
\end{Verbatim}


    \begin{Verbatim}[commandchars=\\\{\}]
{\color{incolor}In [{\color{incolor}2}]:} \PY{n}{Image}\PY{p}{(}\PY{n}{filename} \PY{o}{=} \PY{l+s+s2}{\PYZdq{}}\PY{l+s+s2}{C:/Users/User/Desktop/Fund. of Data Analysis I \PYZhy{}  Udacity/WEEK 4 \PYZhy{} Final Project/Rides by hour \PYZhy{} Taylor.png}\PY{l+s+s2}{\PYZdq{}}\PY{p}{,} \PY{n}{width}\PY{o}{=}\PY{l+m+mi}{700}\PY{p}{,} \PY{n}{height}\PY{o}{=}\PY{l+m+mi}{700}\PY{p}{)}
\end{Verbatim}

\texttt{\color{outcolor}Out[{\color{outcolor}2}]:}
    
    \begin{center}
    \adjustimage{max size={0.9\linewidth}{0.9\paperheight}}{output_9_0.png}
    \end{center}
    { \hspace*{\fill} \\}
    

    \textbf{Visualização 2:} Where do people ride Bike Share?

    \begin{Verbatim}[commandchars=\\\{\}]
{\color{incolor}In [{\color{incolor}3}]:} \PY{n}{Image}\PY{p}{(}\PY{l+s+s2}{\PYZdq{}}\PY{l+s+s2}{C:/Users/User/Desktop/Fund. of Data Analysis I \PYZhy{}  Udacity/WEEK 4 \PYZhy{} Final Project/Interactive Chart \PYZhy{} Taylor.png}\PY{l+s+s2}{\PYZdq{}}\PY{p}{)}
\end{Verbatim}

\texttt{\color{outcolor}Out[{\color{outcolor}3}]:}
    
    \begin{center}
    \adjustimage{max size={0.9\linewidth}{0.9\paperheight}}{output_11_0.png}
    \end{center}
    { \hspace*{\fill} \\}
    

    \section{Pergunta 2.1}\label{pergunta-2.1}

Quais são as perguntas que foram respondidas com suas visualizações?
Porque você as escolheu?

    ** Resposta **: As minhas visualizações respondem quando a Bay Area Bike
Share é usada e quais os locais que apresentam maior número de usuário.
A escolha dessas perguntas é motivada pela importância de compreender o
padrão dos usuários, podendo classifica-los, assim como Taylor nos
mostrou, e localizar a estações de maior fluxo, visando aprofundar a
análise nesses usuários para entender o que os motiva a utilizar o
serviço, bem como entender o motivo pelo qual algumas estações quase não
são utilizadas.

    \subsection{Data Wrangling (Limpeza de
Dados)}\label{data-wrangling-limpeza-de-dados}

Agora é a sua vez de explorar os dados. Os
\href{http://www.bayareabikeshare.com/open-data}{dados abertos} do Ano 1
e do Ano 2 da página Bay Area Bike Share já foram fornecidos com os
materiais do projeto; você não precisa baixar nada extra. O dado vem em
três partes: a primeira metade do Ano 1 (arquivos a partir de
\texttt{201402}), a segunda metade do Ano 1 (arquivos a partir de
\texttt{201408}) e todo o Ano 2 (arquivos a partir de \texttt{201508}).
Existem três arquivos de dados principais associados a cada parte: dados
de viagem que mostram informações sobre cada viagem no sistema
(\texttt{*\_trip\_data.csv}), informações sobre as estações no sistema
(\texttt{*\_station\_data.csv}) e dados meteorológicos diários para cada
cidade no sistema (\texttt{*\_weather\_data.csv}).

Ao lidar com muitos dados, pode ser útil começar trabalhando com apenas
uma amostra dos dados. Desta forma, será muito mais fácil verificar se
nossos passos da limpeza de dados (Data Wrangling) estão funcionando,
pois nosso código demorará menos tempo para ser concluído. Uma vez que
estamos satisfeitos com a forma como funcionam as coisas, podemos
configurar o processo para trabalhar no conjunto de dados como um todo.

Uma vez que a maior parte dos dados está contida na informação de
viagem, devemos segmentar a procura de um subconjunto dos dados da
viagem para nos ajudar a seguir em frente. Você começará olhando apenas
o primeiro mês dos dados da viagem de bicicleta, de 2013-08-29 a
2013-09-30. O código abaixo selecionará os dados da primeira metade do
primeiro ano, então escreverá o valor do primeiro mês de dados para um
arquivo de saída. Este código explora o fato de que os dados são
classificados por data (note que os dois primeiros dias são
classificados por tempo de viagem, em vez de serem completamente
cronológicos).

Primeiro, carregue todos os pacotes e funções que você usará em sua
análise executando a primeira célula de código abaixo. Em seguida,
execute a segunda célula de código para ler um subconjunto do primeiro
arquivo de dados de viagem e escrever um novo arquivo contendo apenas o
subconjunto em que inicialmente estamos interessados.

\begin{quote}
\textbf{Dica}: Você pode executar uma célula de código ou renderizar um
texto em Markdown clicando na célula e usando o atalho do teclado
\textbf{Shift} + \textbf{Enter} ou \textbf{Shift} + \textbf{Return}.
Alternativamente, uma célula de código pode ser executada usando o botão
\textbf{Play} na barra de ferramentas (a cima no IPython Notebook)
depois de selecioná-la. Enquanto a célula estiver em execução, você verá
um asterisco na mensagem à esquerda da célula, ou seja,
\texttt{In\ {[}*{]}:}. O asterisco mudará para um número para mostrar
que a execução foi concluída, Ex: \texttt{In\ {[}1{]}}. Se houver saída,
ele aparecerá como \texttt{Out\ {[}1{]}:}, com um número apropriado para
coincidir com o número de "In".
\end{quote}

    \begin{Verbatim}[commandchars=\\\{\}]
{\color{incolor}In [{\color{incolor}4}]:} \PY{c+c1}{\PYZsh{} Importa todas as bibliotecas necessárias}
        \PY{o}{\PYZpc{}}\PY{k}{matplotlib} inline
        \PY{k+kn}{import} \PY{n+nn}{csv}
        \PY{k+kn}{from} \PY{n+nn}{datetime} \PY{k+kn}{import} \PY{n}{datetime}
        \PY{k+kn}{import} \PY{n+nn}{numpy} \PY{k+kn}{as} \PY{n+nn}{np}
        \PY{k+kn}{import} \PY{n+nn}{pandas} \PY{k+kn}{as} \PY{n+nn}{pd}
        
        \PY{k+kn}{from} \PY{n+nn}{babs\PYZus{}datacheck} \PY{k+kn}{import} \PY{n}{question\PYZus{}3}
        \PY{k+kn}{from} \PY{n+nn}{babs\PYZus{}visualizations} \PY{k+kn}{import} \PY{n}{usage\PYZus{}stats}\PY{p}{,} \PY{n}{usage\PYZus{}plot}
        \PY{k+kn}{from} \PY{n+nn}{IPython.display} \PY{k+kn}{import} \PY{n}{display}
\end{Verbatim}


    \paragraph{OBS: Segue abaixo dois link que me ajudaram na
análise}\label{obs-segue-abaixo-dois-link-que-me-ajudaram-na-anuxe1lise}

http://thfield.github.io/babs/index.html

https://review.udacity.com/\#!/rubrics/1071/view

    \begin{Verbatim}[commandchars=\\\{\}]
{\color{incolor}In [{\color{incolor}5}]:} \PY{c+c1}{\PYZsh{} definição dos arquivos}
        
        \PY{n}{file\PYZus{}in}  \PY{o}{=} \PY{l+s+s1}{\PYZsq{}}\PY{l+s+s1}{201402\PYZus{}trip\PYZus{}data.csv}\PY{l+s+s1}{\PYZsq{}}
        \PY{n}{file\PYZus{}out} \PY{o}{=} \PY{l+s+s1}{\PYZsq{}}\PY{l+s+s1}{201309\PYZus{}trip\PYZus{}data.csv}\PY{l+s+s1}{\PYZsq{}} \PY{c+c1}{\PYZsh{}Inicialmente é um arquivo em branco}
        
        \PY{c+c1}{\PYZsh{}Abre o arquivo file\PYZus{}out para edição e o arquivo file\PYZus{}in para leitura}
        \PY{k}{with} \PY{n+nb}{open}\PY{p}{(}\PY{n}{file\PYZus{}out}\PY{p}{,} \PY{l+s+s1}{\PYZsq{}}\PY{l+s+s1}{w}\PY{l+s+s1}{\PYZsq{}}\PY{p}{)} \PY{k}{as} \PY{n}{f\PYZus{}out}\PY{p}{,} \PY{n+nb}{open}\PY{p}{(}\PY{n}{file\PYZus{}in}\PY{p}{,} \PY{l+s+s1}{\PYZsq{}}\PY{l+s+s1}{r}\PY{l+s+s1}{\PYZsq{}}\PY{p}{)} \PY{k}{as} \PY{n}{f\PYZus{}in}\PY{p}{:}
            \PY{c+c1}{\PYZsh{}configura o leitor de csv}
            \PY{n}{in\PYZus{}reader} \PY{o}{=} \PY{n}{csv}\PY{o}{.}\PY{n}{reader}\PY{p}{(}\PY{n}{f\PYZus{}in}\PY{p}{)}
            \PY{n}{out\PYZus{}writer} \PY{o}{=} \PY{n}{csv}\PY{o}{.}\PY{n}{writer}\PY{p}{(}\PY{n}{f\PYZus{}out}\PY{p}{)}
        
            \PY{c+c1}{\PYZsh{} escreve os dados no arquivo de saída até que a data limite seja atingida}
            \PY{k}{while} \PY{n+nb+bp}{True}\PY{p}{:}
                
                \PY{c+c1}{\PYZsh{}For acess the next row of data}
                \PY{n}{datarow} \PY{o}{=} \PY{n+nb}{next}\PY{p}{(}\PY{n}{in\PYZus{}reader}\PY{p}{)}
                
                \PY{c+c1}{\PYZsh{} data de início de das viagens na terceira coluna no formato \PYZsq{}m/d/yyyy HH:MM\PYZsq{} = sting}
                \PY{c+c1}{\PYZsh{}acessando a coluna da data[2] e coletando do inicio até a posição 8 (inclusa), ou seja, [:9]}
                \PY{k}{if} \PY{n}{datarow}\PY{p}{[}\PY{l+m+mi}{2}\PY{p}{]}\PY{p}{[}\PY{p}{:}\PY{l+m+mi}{9}\PY{p}{]} \PY{o}{==} \PY{l+s+s1}{\PYZsq{}}\PY{l+s+s1}{10/1/2013}\PY{l+s+s1}{\PYZsq{}}\PY{p}{:} 
                    \PY{k}{break}
                \PY{n}{out\PYZus{}writer}\PY{o}{.}\PY{n}{writerow}\PY{p}{(}\PY{n}{datarow}\PY{p}{)}
\end{Verbatim}


    O arquivo '201309\_trip\_data.csv', antes vazio, agora contém os dados
do primeiro mes da parte trip\_data do primeiro ano.

    \subsubsection{Condensando os Dados de
Viagem}\label{condensando-os-dados-de-viagem}

O primeiro passo é analisar a estrutura do conjunto de dados para ver se
há alguma limpeza de dados que devemos realizar. A célula abaixo irá ler
o arquivo de dados amostrado que você criou na célula anterior. Você
deve imprimir as primeiras linhas da tabela.

    \begin{Verbatim}[commandchars=\\\{\}]
{\color{incolor}In [{\color{incolor}6}]:} \PY{n}{sample\PYZus{}data} \PY{o}{=} \PY{n}{pd}\PY{o}{.}\PY{n}{read\PYZus{}csv}\PY{p}{(}\PY{l+s+s1}{\PYZsq{}}\PY{l+s+s1}{201309\PYZus{}trip\PYZus{}data.csv}\PY{l+s+s1}{\PYZsq{}}\PY{p}{)}
        
        \PY{c+c1}{\PYZsh{} TODO: escreva o código para visualizar as primeiras linhas}
        \PY{n}{sample\PYZus{}data}\PY{o}{.}\PY{n}{head}\PY{p}{(}\PY{p}{)}
\end{Verbatim}


\begin{Verbatim}[commandchars=\\\{\}]
{\color{outcolor}Out[{\color{outcolor}6}]:}    Trip ID  Duration       Start Date             Start Station  \textbackslash{}
        0     4576        63  8/29/2013 14:13  South Van Ness at Market   
        1     4607        70  8/29/2013 14:42        San Jose City Hall   
        2     4130        71  8/29/2013 10:16   Mountain View City Hall   
        3     4251        77  8/29/2013 11:29        San Jose City Hall   
        4     4299        83  8/29/2013 12:02  South Van Ness at Market   
        
           Start Terminal         End Date               End Station  End Terminal  \textbackslash{}
        0              66  8/29/2013 14:14  South Van Ness at Market            66   
        1              10  8/29/2013 14:43        San Jose City Hall            10   
        2              27  8/29/2013 10:17   Mountain View City Hall            27   
        3              10  8/29/2013 11:30        San Jose City Hall            10   
        4              66  8/29/2013 12:04            Market at 10th            67   
        
           Bike \# Subscription Type Zip Code  
        0     520        Subscriber    94127  
        1     661        Subscriber    95138  
        2      48        Subscriber    97214  
        3      26        Subscriber    95060  
        4     319        Subscriber    94103  
\end{Verbatim}
            
    Nesta exploração, vamos nos concentrar nos fatores dos dados da viagem
que afetam o número de viagens realizadas. Vamos focar em algumas
colunas selecionadas: a duração da viagem (trip duration), hora de
início (start time), terminal inicial (start terminal), terminal final
(end terminal) e tipo de assinatura (subscription type). O start time
\textbf{(NÃO SERIA START DATE??)} será dividido em componentes de ano,
mês e hora \textbf{(DIVIDIDO EM ANO, MES, DIA E HORA??)}. Também
adicionaremos uma coluna para o dia da semana e resumiremos o terminal
inicial e final para ser a \emph{cidade} de início e fim.

Vamos primeiro abordar a última parte do processo de limpeza. Execute a
célula de código abaixo para ver como as informações da estação estão
estruturadas e observe como o código criará o mapeamento estação-cidade.
Observe que o mapeamento da estação está configurado como uma função,
\texttt{create\_station\_mapping()}. Uma vez que é possível que mais
estações sejam adicionadas ou removidas ao longo do tempo, esta função
nos permitirá combinar as informações da estação em todas as três partes
dos nossos dados quando estivermos prontos para explorar tudo.

    \begin{Verbatim}[commandchars=\\\{\}]
{\color{incolor}In [{\color{incolor}7}]:} \PY{c+c1}{\PYZsh{} Mostra as primeiras linhas do arquivo de dados das estações}
        \PY{n}{station\PYZus{}info} \PY{o}{=} \PY{n}{pd}\PY{o}{.}\PY{n}{read\PYZus{}csv}\PY{p}{(}\PY{l+s+s1}{\PYZsq{}}\PY{l+s+s1}{201402\PYZus{}station\PYZus{}data.csv}\PY{l+s+s1}{\PYZsq{}}\PY{p}{)}
        \PY{n}{station\PYZus{}info}\PY{o}{.}\PY{n}{head}\PY{p}{(}\PY{l+m+mi}{10}\PY{p}{)}
\end{Verbatim}


\begin{Verbatim}[commandchars=\\\{\}]
{\color{outcolor}Out[{\color{outcolor}7}]:}    station\_id                               name        lat        long  \textbackslash{}
        0           2  San Jose Diridon Caltrain Station  37.329732 -121.901782   
        1           3              San Jose Civic Center  37.330698 -121.888979   
        2           4             Santa Clara at Almaden  37.333988 -121.894902   
        3           5                   Adobe on Almaden  37.331415 -121.893200   
        4           6                   San Pedro Square  37.336721 -121.894074   
        5           7               Paseo de San Antonio  37.333798 -121.886943   
        6           8                San Salvador at 1st  37.330165 -121.885831   
        7           9                          Japantown  37.348742 -121.894715   
        8          10                 San Jose City Hall  37.337391 -121.886995   
        9          11                        MLK Library  37.335885 -121.885660   
        
           dockcount  landmark installation  
        0         27  San Jose     8/6/2013  
        1         15  San Jose     8/5/2013  
        2         11  San Jose     8/6/2013  
        3         19  San Jose     8/5/2013  
        4         15  San Jose     8/7/2013  
        5         15  San Jose     8/7/2013  
        6         15  San Jose     8/5/2013  
        7         15  San Jose     8/5/2013  
        8         15  San Jose     8/6/2013  
        9         19  San Jose     8/6/2013  
\end{Verbatim}
            
    \textbf{NOTA SOBRE OS DADOS DAS ESTAÇÕES:} Embora as estações tenham
sido instaladas antes de 8/29/13 (lançamento do sistema), nenhuma
estação estava ativa até a data de lançamento. Portanto, para capturar
com precisão a popularidade da estação, recomendamos ajustar todas as
datas de instalação pré-lançamento em 29/8/13. \emph{(Ainda não foi
realizado)}

    Preencha a função abaixo de forma que a função retorne um mapeamento
entre o id da estação (\texttt{station\_id}) e a cidade em que ela se
encontra (\texttt{landmark}).

    \begin{Verbatim}[commandchars=\\\{\}]
{\color{incolor}In [{\color{incolor}8}]:} \PY{c+c1}{\PYZsh{} esta função será usada mais tarde para criar o mapeamento entre station e cidade}
        \PY{k}{def} \PY{n+nf}{create\PYZus{}station\PYZus{}mapping}\PY{p}{(}\PY{n}{station\PYZus{}data}\PY{p}{)}\PY{p}{:}
            \PY{l+s+sd}{\PYZdq{}\PYZdq{}\PYZdq{}}
        \PY{l+s+sd}{    Cria um mapeamento (tambémm conhecido como de\PYZhy{}para) entre a estação }
        \PY{l+s+sd}{    e a cidade}
        \PY{l+s+sd}{    \PYZdq{}\PYZdq{}\PYZdq{}}
            \PY{c+c1}{\PYZsh{} TODO: Inicie esta variável de maneira correta.}
            \PY{n}{station\PYZus{}map} \PY{o}{=} \PY{p}{\PYZob{}}\PY{p}{\PYZcb{}}
            \PY{k}{for} \PY{n}{data\PYZus{}file} \PY{o+ow}{in} \PY{n}{station\PYZus{}data}\PY{p}{:}
                \PY{k}{with} \PY{n+nb}{open}\PY{p}{(}\PY{n}{data\PYZus{}file}\PY{p}{,} \PY{l+s+s1}{\PYZsq{}}\PY{l+s+s1}{r}\PY{l+s+s1}{\PYZsq{}}\PY{p}{)} \PY{k}{as} \PY{n}{f\PYZus{}in}\PY{p}{:}
                    \PY{c+c1}{\PYZsh{} configura o objeto csv reader \PYZhy{} note que está sendo usado o DictReader,}
                    \PY{c+c1}{\PYZsh{} que usa a primeira linha do arquivo como cabeçalho e cria as chaves}
                    \PY{c+c1}{\PYZsh{} do dicionário com estes valores.}
                    \PY{n}{weather\PYZus{}reader} \PY{o}{=} \PY{n}{csv}\PY{o}{.}\PY{n}{DictReader}\PY{p}{(}\PY{n}{f\PYZus{}in}\PY{p}{)}
        
                    \PY{k}{for} \PY{n}{row} \PY{o+ow}{in} \PY{n}{weather\PYZus{}reader}\PY{p}{:} \PY{c+c1}{\PYZsh{}row irá percorrer todas as keys do disct weather\PYZus{}reader}
                        \PY{n}{station\PYZus{}id} \PY{o}{=} \PY{n}{row}\PY{p}{[}\PY{l+s+s1}{\PYZsq{}}\PY{l+s+s1}{station\PYZus{}id}\PY{l+s+s1}{\PYZsq{}}\PY{p}{]}
                        \PY{n}{city} \PY{o}{=} \PY{n}{row}\PY{p}{[}\PY{l+s+s1}{\PYZsq{}}\PY{l+s+s1}{landmark}\PY{l+s+s1}{\PYZsq{}}\PY{p}{]}
                        \PY{n}{station\PYZus{}map}\PY{p}{[}\PY{n}{station\PYZus{}id}\PY{p}{]} \PY{o}{=} \PY{n}{city}
            \PY{k}{return} \PY{n}{station\PYZus{}map}
\end{Verbatim}


    \begin{Verbatim}[commandchars=\\\{\}]
{\color{incolor}In [{\color{incolor}9}]:} \PY{c+c1}{\PYZsh{}\PYZsh{}\PYZsh{} JUST FOR TEST}
        
        \PY{c+c1}{\PYZsh{}test = [\PYZsq{}201402\PYZus{}station\PYZus{}data.csv\PYZsq{}]}
        \PY{c+c1}{\PYZsh{}print create\PYZus{}station\PYZus{}mapping(test)}
\end{Verbatim}


    Você pode agora usar o mapeamento para condensar as viagens para as
colunas selecionadas acima. Isto acontecerá na função abaixo
\texttt{summarise\_data()}. Nela o módulo \texttt{datetime} é usado para
fazer o \textbf{p}arse do tempo (timestamp) em formato de strings no
arquivo original para um objeto usando a função \texttt{strptime}. Este
objeto permitirá a conversão para outros \textbf{f}ormatos de datas
usando a função \texttt{strftime}. O objeto possui também outras funções
que facilitam a manipulação dos dados. Veja
\href{http://usandopython.com.br/manipulando-data-hora-python-datetime/}{este
tutorial} para entender um pouco melhor como trabalhar com a biblioteca.

Você precisa concluir duas tarefas para completar a função
\texttt{summarise\_data()}. Inicialmente, você deverá realizar a
operação de converter a duração das viagens de segundos para minutos.
Esta é muito fácil, pois existem 60 segundos em um minuto!

Na sequência, você deve criar colunas para o ano, mês, hora e dia da
semana. Verifique o tutorial acima ou a
\href{https://docs.python.org/2/library/datetime.html\#datetime-objects}{documentação
para o objeto de datetime no módulo datetime}.

** TODO: Encontre os atributos e métodos necessários para poder
completar o código abaixo **

\emph{Dica}: Você pode abrir uma nova caixa para testar um pedaço do
código ou verificar uma variável que seja global. Caso ela esteja dentro
da função, você também pode usar o comando \texttt{print()} para
imprimi-la e ajudar no Debug.

    \begin{Verbatim}[commandchars=\\\{\}]
{\color{incolor}In [{\color{incolor}10}]:} \PY{c+c1}{\PYZsh{}colunas selecionadas: a duração da viagem (trip duration), hora de início (start time), }
         \PY{c+c1}{\PYZsh{}terminal inicial (start terminal), terminal final (end terminal) e tipo de assinatura (subscription type). }
         \PY{c+c1}{\PYZsh{}O start time **(NÃO SERIA START DATE??)** será dividido em componentes de ano, mês e hora **(DIVIDIDO EM ANO, MES, DIA E HORA??)**. }
         \PY{c+c1}{\PYZsh{}Também adicionaremos uma coluna para o dia da semana e resumiremos o terminal inicial e final para ser a \PYZus{}cidade\PYZus{} de}
         \PY{c+c1}{\PYZsh{}início e fim.}
         
         \PY{k}{def} \PY{n+nf}{summarise\PYZus{}data}\PY{p}{(}\PY{n}{trip\PYZus{}in}\PY{p}{,} \PY{n}{station\PYZus{}data}\PY{p}{,} \PY{n}{trip\PYZus{}out}\PY{p}{)}\PY{p}{:}
             \PY{l+s+sd}{\PYZdq{}\PYZdq{}\PYZdq{}}
         \PY{l+s+sd}{    Esta função recebe informações de viagem e estação e produz um novo}
         \PY{l+s+sd}{    arquivo de dados com um resumo condensado das principais informações de viagem.Os }
         \PY{l+s+sd}{    argumentos trip\PYZus{}in e station\PYZus{}data serão listas de arquivos de dados para}
         \PY{l+s+sd}{    as informações da viagem e da estação enquanto trip\PYZus{}out especifica o local}
         \PY{l+s+sd}{    para o qual os dados sumarizados serão escritos.}
         \PY{l+s+sd}{    \PYZdq{}\PYZdq{}\PYZdq{}}
             \PY{c+c1}{\PYZsh{} gera o dicionário de mapeamento entre estações e cidades}
             \PY{n}{station\PYZus{}map} \PY{o}{=} \PY{n}{create\PYZus{}station\PYZus{}mapping}\PY{p}{(}\PY{n}{station\PYZus{}data}\PY{p}{)}
             
             \PY{k}{with} \PY{n+nb}{open}\PY{p}{(}\PY{n}{trip\PYZus{}out}\PY{p}{,} \PY{l+s+s1}{\PYZsq{}}\PY{l+s+s1}{w}\PY{l+s+s1}{\PYZsq{}}\PY{p}{)} \PY{k}{as} \PY{n}{f\PYZus{}out}\PY{p}{:}
                 \PY{c+c1}{\PYZsh{} configura o objeto de escrita de csv       }
                 \PY{n}{out\PYZus{}colnames} \PY{o}{=} \PY{p}{[}\PY{l+s+s1}{\PYZsq{}}\PY{l+s+s1}{duration}\PY{l+s+s1}{\PYZsq{}}\PY{p}{,} \PY{l+s+s1}{\PYZsq{}}\PY{l+s+s1}{start\PYZus{}date}\PY{l+s+s1}{\PYZsq{}}\PY{p}{,} \PY{l+s+s1}{\PYZsq{}}\PY{l+s+s1}{start\PYZus{}year}\PY{l+s+s1}{\PYZsq{}}\PY{p}{,}
                                 \PY{l+s+s1}{\PYZsq{}}\PY{l+s+s1}{start\PYZus{}month}\PY{l+s+s1}{\PYZsq{}}\PY{p}{,} \PY{l+s+s1}{\PYZsq{}}\PY{l+s+s1}{start\PYZus{}hour}\PY{l+s+s1}{\PYZsq{}}\PY{p}{,} \PY{l+s+s1}{\PYZsq{}}\PY{l+s+s1}{weekday}\PY{l+s+s1}{\PYZsq{}}\PY{p}{,}
                                 \PY{l+s+s1}{\PYZsq{}}\PY{l+s+s1}{start\PYZus{}city}\PY{l+s+s1}{\PYZsq{}}\PY{p}{,} \PY{l+s+s1}{\PYZsq{}}\PY{l+s+s1}{end\PYZus{}city}\PY{l+s+s1}{\PYZsq{}}\PY{p}{,} \PY{l+s+s1}{\PYZsq{}}\PY{l+s+s1}{subscription\PYZus{}type}\PY{l+s+s1}{\PYZsq{}}\PY{p}{]}        
                 \PY{n}{trip\PYZus{}writer} \PY{o}{=} \PY{n}{csv}\PY{o}{.}\PY{n}{DictWriter}\PY{p}{(}\PY{n}{f\PYZus{}out}\PY{p}{,} \PY{n}{fieldnames} \PY{o}{=} \PY{n}{out\PYZus{}colnames}\PY{p}{)}
                 \PY{n}{trip\PYZus{}writer}\PY{o}{.}\PY{n}{writeheader}\PY{p}{(}\PY{p}{)}
                 
                 \PY{k}{for} \PY{n}{data\PYZus{}file} \PY{o+ow}{in} \PY{n}{trip\PYZus{}in}\PY{p}{:}
                     \PY{k}{with} \PY{n+nb}{open}\PY{p}{(}\PY{n}{data\PYZus{}file}\PY{p}{,} \PY{l+s+s1}{\PYZsq{}}\PY{l+s+s1}{r}\PY{l+s+s1}{\PYZsq{}}\PY{p}{)} \PY{k}{as} \PY{n}{f\PYZus{}in}\PY{p}{:}
                         \PY{c+c1}{\PYZsh{} configura o leitor do csv}
                         \PY{n}{trip\PYZus{}reader} \PY{o}{=} \PY{n}{csv}\PY{o}{.}\PY{n}{DictReader}\PY{p}{(}\PY{n}{f\PYZus{}in}\PY{p}{)}
         
                         \PY{c+c1}{\PYZsh{} processa cada linha lendo uma a uma}
                         \PY{k}{for} \PY{n}{row} \PY{o+ow}{in} \PY{n}{trip\PYZus{}reader}\PY{p}{:}
                             \PY{n}{new\PYZus{}point} \PY{o}{=} \PY{p}{\PYZob{}}\PY{p}{\PYZcb{}}
                             
                             \PY{c+c1}{\PYZsh{} converte a duração de segundos para minutos.}
                             \PY{c+c1}{\PYZsh{}\PYZsh{}\PYZsh{} TODO: Pergunta 3a: Adicione uma operação matemática       \PYZsh{}\PYZsh{}\PYZsh{}}
                             \PY{c+c1}{\PYZsh{}\PYZsh{}\PYZsh{} para converter a duração de segundos para minutos.  \PYZsh{}\PYZsh{}\PYZsh{}}
                             \PY{n}{new\PYZus{}point}\PY{p}{[}\PY{l+s+s1}{\PYZsq{}}\PY{l+s+s1}{duration}\PY{l+s+s1}{\PYZsq{}}\PY{p}{]} \PY{o}{=} \PY{n+nb}{float}\PY{p}{(}\PY{n}{row}\PY{p}{[}\PY{l+s+s1}{\PYZsq{}}\PY{l+s+s1}{Duration}\PY{l+s+s1}{\PYZsq{}}\PY{p}{]}\PY{p}{)}\PY{o}{/}\PY{n+nb}{float}\PY{p}{(}\PY{l+m+mi}{60}\PY{p}{)}
                             
                             \PY{c+c1}{\PYZsh{} reformate strings com datas para múltiplas colunas}
                             \PY{c+c1}{\PYZsh{}\PYZsh{}\PYZsh{} TODO: Pergunta 3b: Preencha os \PYZus{}\PYZus{} abaixo para criar os        \PYZsh{}\PYZsh{}\PYZsh{}}
                             \PY{c+c1}{\PYZsh{}\PYZsh{}\PYZsh{} campos experados nas colunas (olhe pelo nome da coluna) \PYZsh{}\PYZsh{}\PYZsh{}}
                             \PY{n}{trip\PYZus{}date} \PY{o}{=} \PY{n}{datetime}\PY{o}{.}\PY{n}{strptime}\PY{p}{(}\PY{n}{row}\PY{p}{[}\PY{l+s+s1}{\PYZsq{}}\PY{l+s+s1}{Start Date}\PY{l+s+s1}{\PYZsq{}}\PY{p}{]}\PY{p}{,} \PY{l+s+s1}{\PYZsq{}}\PY{l+s+s1}{\PYZpc{}}\PY{l+s+s1}{m/}\PY{l+s+si}{\PYZpc{}d}\PY{l+s+s1}{/}\PY{l+s+s1}{\PYZpc{}}\PY{l+s+s1}{Y }\PY{l+s+s1}{\PYZpc{}}\PY{l+s+s1}{H:}\PY{l+s+s1}{\PYZpc{}}\PY{l+s+s1}{M}\PY{l+s+s1}{\PYZsq{}}\PY{p}{)}
                             \PY{n}{new\PYZus{}point}\PY{p}{[}\PY{l+s+s1}{\PYZsq{}}\PY{l+s+s1}{start\PYZus{}date}\PY{l+s+s1}{\PYZsq{}}\PY{p}{]}  \PY{o}{=} \PY{n}{trip\PYZus{}date}\PY{o}{.}\PY{n}{strftime}\PY{p}{(}\PY{l+s+s1}{\PYZsq{}}\PY{l+s+si}{\PYZpc{}d}\PY{l+s+s1}{/}\PY{l+s+s1}{\PYZpc{}}\PY{l+s+s1}{m/}\PY{l+s+s1}{\PYZpc{}}\PY{l+s+s1}{Y}\PY{l+s+s1}{\PYZsq{}}\PY{p}{)}
                             \PY{c+c1}{\PYZsh{}print \PYZsq{}1) \PYZsq{}, new\PYZus{}point[\PYZsq{}start\PYZus{}date\PYZsq{}] }
                             \PY{n}{new\PYZus{}point}\PY{p}{[}\PY{l+s+s1}{\PYZsq{}}\PY{l+s+s1}{start\PYZus{}year}\PY{l+s+s1}{\PYZsq{}}\PY{p}{]}  \PY{o}{=} \PY{n}{trip\PYZus{}date}\PY{o}{.}\PY{n}{year}
                             \PY{c+c1}{\PYZsh{}print \PYZsq{}2) \PYZsq{}, new\PYZus{}point[\PYZsq{}start\PYZus{}year\PYZsq{}]}
                             \PY{n}{new\PYZus{}point}\PY{p}{[}\PY{l+s+s1}{\PYZsq{}}\PY{l+s+s1}{start\PYZus{}month}\PY{l+s+s1}{\PYZsq{}}\PY{p}{]} \PY{o}{=} \PY{n}{trip\PYZus{}date}\PY{o}{.}\PY{n}{month}
                             \PY{c+c1}{\PYZsh{}print \PYZsq{}3) \PYZsq{}, new\PYZus{}point[\PYZsq{}start\PYZus{}month\PYZsq{}]}
                             \PY{n}{new\PYZus{}point}\PY{p}{[}\PY{l+s+s1}{\PYZsq{}}\PY{l+s+s1}{start\PYZus{}hour}\PY{l+s+s1}{\PYZsq{}}\PY{p}{]}  \PY{o}{=} \PY{n}{trip\PYZus{}date}\PY{o}{.}\PY{n}{strftime}\PY{p}{(}\PY{l+s+s1}{\PYZsq{}}\PY{l+s+s1}{\PYZpc{}}\PY{l+s+s1}{H}\PY{l+s+s1}{\PYZsq{}}\PY{p}{)}
                             \PY{c+c1}{\PYZsh{}print \PYZsq{}4) \PYZsq{}, new\PYZus{}point[\PYZsq{}start\PYZus{}hour\PYZsq{}]}
                             \PY{n}{new\PYZus{}point}\PY{p}{[}\PY{l+s+s1}{\PYZsq{}}\PY{l+s+s1}{weekday}\PY{l+s+s1}{\PYZsq{}}\PY{p}{]}     \PY{o}{=} \PY{n}{trip\PYZus{}date}\PY{o}{.}\PY{n}{weekday}\PY{p}{(}\PY{p}{)}
                             \PY{c+c1}{\PYZsh{}print \PYZsq{}5) \PYZsq{}, new\PYZus{}point[\PYZsq{}weekday\PYZsq{}]}
         
                             \PY{c+c1}{\PYZsh{} TODO: mapeia o terminal de inicio e fim com o a cidade de inicio e fim}
                             \PY{n}{new\PYZus{}point}\PY{p}{[}\PY{l+s+s1}{\PYZsq{}}\PY{l+s+s1}{start\PYZus{}city}\PY{l+s+s1}{\PYZsq{}}\PY{p}{]} \PY{o}{=}  \PY{n}{station\PYZus{}map}\PY{p}{[}\PY{n}{row}\PY{p}{[}\PY{l+s+s1}{\PYZsq{}}\PY{l+s+s1}{Start Terminal}\PY{l+s+s1}{\PYZsq{}}\PY{p}{]}\PY{p}{]}
                             \PY{c+c1}{\PYZsh{}print \PYZsq{}6) \PYZsq{}, new\PYZus{}point[\PYZsq{}start\PYZus{}city\PYZsq{}]}
                             \PY{n}{new\PYZus{}point}\PY{p}{[}\PY{l+s+s1}{\PYZsq{}}\PY{l+s+s1}{end\PYZus{}city}\PY{l+s+s1}{\PYZsq{}}\PY{p}{]} \PY{o}{=}  \PY{n}{station\PYZus{}map}\PY{p}{[}\PY{n}{row}\PY{p}{[}\PY{l+s+s1}{\PYZsq{}}\PY{l+s+s1}{End Terminal}\PY{l+s+s1}{\PYZsq{}}\PY{p}{]}\PY{p}{]}
                             \PY{c+c1}{\PYZsh{}print \PYZsq{}7) \PYZsq{}, new\PYZus{}point[\PYZsq{}end\PYZus{}city\PYZsq{}]}
                             \PY{c+c1}{\PYZsh{} TODO: existem dois nomes diferentes para o mesmo campo. Trate cada um deles.}
                             \PY{c+c1}{\PYZsh{}o arquivo csv 201402\PYZus{}trip\PYZus{}data tem \PYZsq{}Subscription\PYZus{}type\PYZsq{} e o 201408 tem \PYZsq{}Subscriber type\PYZsq{} }
                             \PY{k}{if} \PY{l+s+s1}{\PYZsq{}}\PY{l+s+s1}{Subscription Type}\PY{l+s+s1}{\PYZsq{}} \PY{o+ow}{in} \PY{n}{row}\PY{p}{:}
                                 \PY{n}{new\PYZus{}point}\PY{p}{[}\PY{l+s+s1}{\PYZsq{}}\PY{l+s+s1}{subscription\PYZus{}type}\PY{l+s+s1}{\PYZsq{}}\PY{p}{]} \PY{o}{=} \PY{n}{row}\PY{p}{[}\PY{l+s+s1}{\PYZsq{}}\PY{l+s+s1}{Subscription Type}\PY{l+s+s1}{\PYZsq{}}\PY{p}{]}
                             \PY{k}{else}\PY{p}{:}
                                 \PY{n}{new\PYZus{}point}\PY{p}{[}\PY{l+s+s1}{\PYZsq{}}\PY{l+s+s1}{subscription\PYZus{}type}\PY{l+s+s1}{\PYZsq{}}\PY{p}{]} \PY{o}{=} \PY{n}{row}\PY{p}{[}\PY{l+s+s1}{\PYZsq{}}\PY{l+s+s1}{Subscriber Type}\PY{l+s+s1}{\PYZsq{}}\PY{p}{]}
         
                             \PY{c+c1}{\PYZsh{} escreve a informação processada para o arquivo de saída.}
                             \PY{n}{trip\PYZus{}writer}\PY{o}{.}\PY{n}{writerow}\PY{p}{(}\PY{n}{new\PYZus{}point}\PY{p}{)}
\end{Verbatim}


    \section{Pergunta 3:}\label{pergunta-3}

Execute o bloco de código abaixo para chamar a função
\texttt{summarise\_data()} que você terminou na célula acima. Ela usará
os dados contidos nos arquivos listados nas variáveis \texttt{trip\_in}
e \texttt{station\_data} e escreverá um novo arquivo no local
especificado na variável \texttt{trip\_out}. Se você executou a limpeza
de dados corretamente, o bloco de código abaixo imprimirá as primeiras
linhas do DataFrame e uma mensagem que verificando se as contagens de
dados estão corretas.

    \begin{Verbatim}[commandchars=\\\{\}]
{\color{incolor}In [{\color{incolor}11}]:} \PY{c+c1}{\PYZsh{} processe os dados usando a função criada acima}
         \PY{n}{station\PYZus{}data} \PY{o}{=} \PY{p}{[}\PY{l+s+s1}{\PYZsq{}}\PY{l+s+s1}{201402\PYZus{}station\PYZus{}data.csv}\PY{l+s+s1}{\PYZsq{}}\PY{p}{]}
         \PY{n}{trip\PYZus{}in} \PY{o}{=} \PY{p}{[}\PY{l+s+s1}{\PYZsq{}}\PY{l+s+s1}{201309\PYZus{}trip\PYZus{}data.csv}\PY{l+s+s1}{\PYZsq{}}\PY{p}{]}
         \PY{n}{trip\PYZus{}out} \PY{o}{=} \PY{l+s+s1}{\PYZsq{}}\PY{l+s+s1}{201309\PYZus{}trip\PYZus{}summary.csv}\PY{l+s+s1}{\PYZsq{}}
         \PY{n}{summarise\PYZus{}data}\PY{p}{(}\PY{n}{trip\PYZus{}in}\PY{p}{,} \PY{n}{station\PYZus{}data}\PY{p}{,} \PY{n}{trip\PYZus{}out}\PY{p}{)}
\end{Verbatim}


    \begin{Verbatim}[commandchars=\\\{\}]
{\color{incolor}In [{\color{incolor}12}]:} \PY{c+c1}{\PYZsh{} Carregue os dados novamente mostrando os dados}
         \PY{c+c1}{\PYZsh{}\PYZsh{} TODO: Complete o código para leitura dos dados no arquivo criado na função acima}
         \PY{n}{sample\PYZus{}data} \PY{o}{=} \PY{n}{pd}\PY{o}{.}\PY{n}{read\PYZus{}csv}\PY{p}{(}\PY{n}{trip\PYZus{}out}\PY{p}{)}
         \PY{n}{display}\PY{p}{(}\PY{n}{sample\PYZus{}data}\PY{o}{.}\PY{n}{head}\PY{p}{(}\PY{p}{)}\PY{p}{)}
\end{Verbatim}


    
    \begin{verbatim}
   duration  start_date  start_year  start_month  start_hour  weekday  \
0  1.050000  29/08/2013        2013            8          14        3   
1  1.166667  29/08/2013        2013            8          14        3   
2  1.183333  29/08/2013        2013            8          10        3   
3  1.283333  29/08/2013        2013            8          11        3   
4  1.383333  29/08/2013        2013            8          12        3   

      start_city       end_city subscription_type  
0  San Francisco  San Francisco        Subscriber  
1       San Jose       San Jose        Subscriber  
2  Mountain View  Mountain View        Subscriber  
3       San Jose       San Jose        Subscriber  
4  San Francisco  San Francisco        Subscriber  
    \end{verbatim}

    
    \begin{Verbatim}[commandchars=\\\{\}]
{\color{incolor}In [{\color{incolor}13}]:} \PY{c+c1}{\PYZsh{} Verifica o DataFrame contando o número de pontos de dados com as características de }
         \PY{c+c1}{\PYZsh{} tempo corretas.}
         \PY{n}{question\PYZus{}3}\PY{p}{(}\PY{n}{sample\PYZus{}data}\PY{p}{)}
\end{Verbatim}


    \begin{Verbatim}[commandchars=\\\{\}]
Todas as contagens estão como esperadas.

    \end{Verbatim}

    \begin{quote}
\textbf{Dica}: se você salvar um notebook do jupyter, a saída dos blocos
de código em execução também será salva. No entanto, o estado do seu
arquivo será reiniciado uma vez que uma nova sessão será iniciada.
Certifique-se de que você execute todos os blocos de código necessários
da sessão anterior para restabelecer variáveis e funções antes de
continuar de onde você deixou na última vez.
\end{quote}

    \subsection{Análise Exploratória de
Dados}\label{anuxe1lise-exploratuxf3ria-de-dados}

Agora que você tem alguns dados salvos em um arquivo, vejamos algumas
tendências iniciais nos dados. Algum código já foi escrito para você no
script \url{babs_visualizations.py} para ajudar a resumir e visualizar
os dados; Isso foi importado como as funções \texttt{usage\_stats()} e
\texttt{usage\_plot()}. Nesta seção, vamos percorrer algumas das coisas
que você pode fazer com as funções, e você usará as funções para você
mesmo na última parte do projeto. Primeiro, execute a seguinte célula
para carregar os dados. Depois preencha a célula abaixo com os comandos
para verificar os dados básicos sobre os dados.

    \begin{Verbatim}[commandchars=\\\{\}]
{\color{incolor}In [{\color{incolor}14}]:} \PY{n}{trip\PYZus{}data} \PY{o}{=} \PY{n}{pd}\PY{o}{.}\PY{n}{read\PYZus{}csv}\PY{p}{(}\PY{l+s+s1}{\PYZsq{}}\PY{l+s+s1}{201309\PYZus{}trip\PYZus{}summary.csv}\PY{l+s+s1}{\PYZsq{}}\PY{p}{)}
\end{Verbatim}


    \begin{Verbatim}[commandchars=\\\{\}]
{\color{incolor}In [{\color{incolor}15}]:} \PY{n+nb}{type}\PY{p}{(}\PY{n}{trip\PYZus{}data}\PY{p}{)}
\end{Verbatim}


\begin{Verbatim}[commandchars=\\\{\}]
{\color{outcolor}Out[{\color{outcolor}15}]:} pandas.core.frame.DataFrame
\end{Verbatim}
            
    \begin{Verbatim}[commandchars=\\\{\}]
{\color{incolor}In [{\color{incolor}16}]:} \PY{n}{trip\PYZus{}data}\PY{o}{.}\PY{n}{head}\PY{p}{(}\PY{p}{)}
\end{Verbatim}


\begin{Verbatim}[commandchars=\\\{\}]
{\color{outcolor}Out[{\color{outcolor}16}]:}    duration  start\_date  start\_year  start\_month  start\_hour  weekday  \textbackslash{}
         0  1.050000  29/08/2013        2013            8          14        3   
         1  1.166667  29/08/2013        2013            8          14        3   
         2  1.183333  29/08/2013        2013            8          10        3   
         3  1.283333  29/08/2013        2013            8          11        3   
         4  1.383333  29/08/2013        2013            8          12        3   
         
               start\_city       end\_city subscription\_type  
         0  San Francisco  San Francisco        Subscriber  
         1       San Jose       San Jose        Subscriber  
         2  Mountain View  Mountain View        Subscriber  
         3       San Jose       San Jose        Subscriber  
         4  San Francisco  San Francisco        Subscriber  
\end{Verbatim}
            
    Para ajudar nessa formatação:
\href{https://mkaz.tech/code/python-string-format-cookbook/}{Python
String Format Cookbook}

    \begin{Verbatim}[commandchars=\\\{\}]
{\color{incolor}In [{\color{incolor}17}]:} \PY{c+c1}{\PYZsh{} TODO: preencha os campos com os dados de acordo com o print}
         \PY{k}{print} \PY{l+s+s1}{\PYZsq{}}\PY{l+s+s1}{Existem \PYZob{}:d\PYZcb{} pontos no conjunto de dados}\PY{l+s+s1}{\PYZsq{}}\PY{o}{.}\PY{n}{format}\PY{p}{(}\PY{n+nb}{len}\PY{p}{(}\PY{n}{trip\PYZus{}data}\PY{o}{.}\PY{n}{index}\PY{p}{)}\PY{p}{)}
         \PY{k}{print} \PY{l+s+s1}{\PYZsq{}}\PY{l+s+s1}{A duração média das viagens foi de \PYZob{}:.2f\PYZcb{} minutos}\PY{l+s+s1}{\PYZsq{}}\PY{o}{.}\PY{n}{format}\PY{p}{(}\PY{n}{np}\PY{o}{.}\PY{n}{mean}\PY{p}{(}\PY{n}{trip\PYZus{}data}\PY{p}{[}\PY{l+s+s1}{\PYZsq{}}\PY{l+s+s1}{duration}\PY{l+s+s1}{\PYZsq{}}\PY{p}{]}\PY{p}{)}\PY{p}{)}
         \PY{k}{print} \PY{l+s+s1}{\PYZsq{}}\PY{l+s+s1}{A mediana das durações das viagens foi de \PYZob{}:.2f\PYZcb{} minutos}\PY{l+s+s1}{\PYZsq{}}\PY{o}{.}\PY{n}{format}\PY{p}{(}\PY{n}{np}\PY{o}{.}\PY{n}{median}\PY{p}{(}\PY{n}{trip\PYZus{}data}\PY{p}{[}\PY{l+s+s1}{\PYZsq{}}\PY{l+s+s1}{duration}\PY{l+s+s1}{\PYZsq{}}\PY{p}{]}\PY{p}{)}\PY{p}{)}
         
         \PY{c+c1}{\PYZsh{} TODO: verificando os quartis}
         \PY{n}{duration\PYZus{}qtiles} \PY{o}{=} \PY{n}{trip\PYZus{}data}\PY{p}{[}\PY{l+s+s1}{\PYZsq{}}\PY{l+s+s1}{duration}\PY{l+s+s1}{\PYZsq{}}\PY{p}{]}\PY{o}{.}\PY{n}{quantile}\PY{p}{(}\PY{p}{[}\PY{o}{.}\PY{l+m+mi}{25}\PY{p}{,} \PY{o}{.}\PY{l+m+mi}{5}\PY{p}{,} \PY{o}{.}\PY{l+m+mi}{75}\PY{p}{]}\PY{p}{)}\PY{o}{.}\PY{n}{as\PYZus{}matrix}\PY{p}{(}\PY{p}{)}
         \PY{c+c1}{\PYZsh{}print duration\PYZus{}qtiles}
         \PY{c+c1}{\PYZsh{}print type(duration\PYZus{}qtiles)}
         \PY{k}{print} \PY{l+s+s1}{\PYZsq{}}\PY{l+s+s1}{25}\PY{l+s+si}{\PYZpc{} d}\PY{l+s+s1}{as viagens foram mais curtas do que \PYZob{}:.2f\PYZcb{} minutos}\PY{l+s+s1}{\PYZsq{}}\PY{o}{.}\PY{n}{format}\PY{p}{(}\PY{n}{duration\PYZus{}qtiles}\PY{p}{[}\PY{l+m+mi}{0}\PY{p}{]}\PY{p}{)}
         \PY{k}{print} \PY{l+s+s1}{\PYZsq{}}\PY{l+s+s1}{25}\PY{l+s+si}{\PYZpc{} d}\PY{l+s+s1}{as viagens foram mais compridas do que \PYZob{}:.2f\PYZcb{} minutos}\PY{l+s+s1}{\PYZsq{}}\PY{o}{.}\PY{n}{format}\PY{p}{(}\PY{n}{duration\PYZus{}qtiles}\PY{p}{[}\PY{l+m+mi}{2}\PY{p}{]}\PY{p}{)}
\end{Verbatim}


    \begin{Verbatim}[commandchars=\\\{\}]
Existem 27345 pontos no conjunto de dados
A duração média das viagens foi de 27.60 minutos
A mediana das durações das viagens foi de 10.72 minutos
25\% das viagens foram mais curtas do que 6.82 minutos
25\% das viagens foram mais compridas do que 17.28 minutos

    \end{Verbatim}

    \begin{Verbatim}[commandchars=\\\{\}]
{\color{incolor}In [{\color{incolor}18}]:} \PY{c+c1}{\PYZsh{} execute este campo para verificar os seu processamento acima.}
         \PY{c+c1}{\PYZsh{}usage\PYZus{}stats(trip\PYZus{}data)}
\end{Verbatim}


    Você deve ver que há mais de 27.000 viagens no primeiro mês e que a
duração média da viagem é maior do que a duração mediana da viagem (o
ponto em que 50\% das viagens são mais curtas e 50\% são mais longas).
Na verdade, a média é maior que as durações de 75\% das viagens mais
curtas. Isso será interessante para ver mais adiante.

Vamos começar a ver como essas viagens são divididas por tipo de
inscrição. Uma maneira fácil de construir uma intuição sobre os dados é
traçá-los.

Lembre-se que o Pandas possui maneiras de plotar os gráficos diretamente
de um DataFrame. Para cada tipo de dados/análises se pode usar um tipo
diferente de gráfico mais apropriado para a análise que se está fazendo.

Na caixa abaixo, faça um gráfico de viagens x tipo de subscrição do tipo
barras.

    \textbf{OBS: }Para conseguir realizar esse plot utilizei:
\href{https://community.modeanalytics.com/python/tutorial/counting-and-plotting-in-python/}{Counting
Values \& Basic Plotting in Python}

    ** Tentativa 1**

    \begin{Verbatim}[commandchars=\\\{\}]
{\color{incolor}In [{\color{incolor}19}]:} \PY{c+c1}{\PYZsh{} TODO: plote um gráfico de barras que mostre quantidade de viagens por subscription\PYZus{}type}
         \PY{c+c1}{\PYZsh{} lembrando que quando o comando .plot é usado, se pode escolher o tipo de gráfico usando }
         \PY{c+c1}{\PYZsh{} o parâmetro kind. Ex: plot(kind=\PYZsq{}bar\PYZsq{})}
         \PY{k}{print} \PY{n}{trip\PYZus{}data}\PY{p}{[}\PY{l+s+s1}{\PYZsq{}}\PY{l+s+s1}{subscription\PYZus{}type}\PY{l+s+s1}{\PYZsq{}}\PY{p}{]}\PY{o}{.}\PY{n}{value\PYZus{}counts}\PY{p}{(}\PY{p}{)}\PY{o}{.}\PY{n}{plot}\PY{p}{(}\PY{n}{kind}\PY{o}{=}\PY{l+s+s1}{\PYZsq{}}\PY{l+s+s1}{bar}\PY{l+s+s1}{\PYZsq{}}\PY{p}{,} \PY{n}{color}\PY{o}{=}\PY{l+s+s1}{\PYZsq{}}\PY{l+s+s1}{g}\PY{l+s+s1}{\PYZsq{}}\PY{l+s+s1}{\PYZsq{}}\PY{l+s+s1}{b}\PY{l+s+s1}{\PYZsq{}}\PY{p}{,} \PY{n}{rot}\PY{o}{=}\PY{l+m+mi}{0}\PY{p}{,} \PY{n}{position} \PY{o}{=}\PY{l+m+mi}{1}\PY{p}{)}
         
         \PY{c+c1}{\PYZsh{}trip\PYZus{}data[\PYZsq{}subscription\PYZus{}type\PYZsq{}].plot(data= kind=\PYZsq{}bar\PYZsq{})}
\end{Verbatim}


    \begin{Verbatim}[commandchars=\\\{\}]
Axes(0.125,0.125;0.775x0.755)

    \end{Verbatim}

    \begin{center}
    \adjustimage{max size={0.9\linewidth}{0.9\paperheight}}{output_44_1.png}
    \end{center}
    { \hspace*{\fill} \\}
    
    Para que você possa conferir se os seus gráficos estão corretos,
usaremos a função \texttt{use\_plot()}. O segundo argumento da função
nos permite contar as viagens em uma variável selecionada, exibindo as
informações em um gráfico. A expressão abaixo mostrará como deve ter
ficado o seu gráfico acima.

    \begin{Verbatim}[commandchars=\\\{\}]
{\color{incolor}In [{\color{incolor}20}]:} \PY{c+c1}{\PYZsh{} como o seu gráfico deve ficar. Descomente a linha abaixo caso queira rodar este comando}
         \PY{c+c1}{\PYZsh{}usage\PYZus{}plot(trip\PYZus{}data, \PYZsq{}subscription\PYZus{}type\PYZsq{})}
\end{Verbatim}


    \begin{quote}
\emph{Nota}: Perceba que provavelmente o seu gráfico não ficou
exatamente igual, principalmente pelo título e pelo nome dos eixos.
Lembre-se, estes são detalhes mas fazem toda a diferença quando você for
apresentar os gráficos que você analisou. Neste Nanodegree não focaremos
nestas questões, mas tenha em mente que ter os gráficos acertados é de
extrema importância.
\end{quote}

    Parece que existe 50\% mais viagens feitas por assinantes (subscribers)
no primeiro mês do que outro tipos de consumidores. Vamos tentar uma
outra variável. Como é a distribuição da duração das viagens (trip
duration)?

    ** Tentativa 1**

    \begin{Verbatim}[commandchars=\\\{\}]
{\color{incolor}In [{\color{incolor}21}]:} \PY{c+c1}{\PYZsh{} TODO: Faça um gráfico baseado nas durações}
         \PY{n}{trip\PYZus{}duration} \PY{o}{=} \PY{n}{trip\PYZus{}data}\PY{p}{[}\PY{l+s+s1}{\PYZsq{}}\PY{l+s+s1}{duration}\PY{l+s+s1}{\PYZsq{}}\PY{p}{]}\PY{o}{.}\PY{n}{plot}\PY{p}{(}\PY{n}{kind}\PY{o}{=}\PY{l+s+s1}{\PYZsq{}}\PY{l+s+s1}{hist}\PY{l+s+s1}{\PYZsq{}}\PY{p}{,} \PY{n}{bins}\PY{o}{=}\PY{l+m+mi}{10}\PY{p}{,} \PY{n+nb}{range}\PY{o}{=}\PY{p}{(}\PY{l+m+mi}{0}\PY{p}{,}\PY{l+m+mi}{10000}\PY{p}{)}\PY{p}{,} \PY{n}{xlim}\PY{o}{=}\PY{p}{[}\PY{l+m+mi}{0}\PY{p}{,}\PY{l+m+mi}{10000}\PY{p}{]}\PY{p}{,} \PY{n}{ylim}\PY{o}{=}\PY{p}{[}\PY{l+m+mi}{0}\PY{p}{,}\PY{l+m+mi}{30000}\PY{p}{]}\PY{p}{,} \PY{n}{title}\PY{o}{=}\PY{l+s+s1}{\PYZsq{}}\PY{l+s+s1}{Trip Duration \PYZhy{} First Month}\PY{l+s+s1}{\PYZsq{}}\PY{p}{)}
         \PY{n}{trip\PYZus{}duration}\PY{o}{.}\PY{n}{set\PYZus{}xlabel}\PY{p}{(}\PY{l+s+s1}{\PYZsq{}}\PY{l+s+s1}{Duration [min]}\PY{l+s+s1}{\PYZsq{}}\PY{p}{)}
         \PY{n}{trip\PYZus{}duration}\PY{o}{.}\PY{n}{set\PYZus{}ylabel}\PY{p}{(}\PY{l+s+s1}{\PYZsq{}}\PY{l+s+s1}{Number of Trip}\PY{l+s+s1}{\PYZsq{}}\PY{p}{)}
\end{Verbatim}


\begin{Verbatim}[commandchars=\\\{\}]
{\color{outcolor}Out[{\color{outcolor}21}]:} <matplotlib.text.Text at 0xbf9b160>
\end{Verbatim}
            
    \begin{center}
    \adjustimage{max size={0.9\linewidth}{0.9\paperheight}}{output_50_1.png}
    \end{center}
    { \hspace*{\fill} \\}
    
    \textbf{Tentativa 2} - Sem defnir limite de eixos e 'bins'

    \begin{Verbatim}[commandchars=\\\{\}]
{\color{incolor}In [{\color{incolor}22}]:} \PY{c+c1}{\PYZsh{} TODO: Faça um gráfico baseado nas durações}
         \PY{n}{trip\PYZus{}duration} \PY{o}{=} \PY{n}{trip\PYZus{}data}\PY{p}{[}\PY{l+s+s1}{\PYZsq{}}\PY{l+s+s1}{duration}\PY{l+s+s1}{\PYZsq{}}\PY{p}{]}\PY{o}{.}\PY{n}{plot}\PY{p}{(}\PY{n}{kind}\PY{o}{=}\PY{l+s+s1}{\PYZsq{}}\PY{l+s+s1}{hist}\PY{l+s+s1}{\PYZsq{}}\PY{p}{,} \PY{n}{title}\PY{o}{=}\PY{l+s+s1}{\PYZsq{}}\PY{l+s+s1}{Trip Duration \PYZhy{} First Month}\PY{l+s+s1}{\PYZsq{}}\PY{p}{)}
         \PY{n}{trip\PYZus{}duration}\PY{o}{.}\PY{n}{set\PYZus{}xlabel}\PY{p}{(}\PY{l+s+s1}{\PYZsq{}}\PY{l+s+s1}{Duration [min]}\PY{l+s+s1}{\PYZsq{}}\PY{p}{)}
         \PY{n}{trip\PYZus{}duration}\PY{o}{.}\PY{n}{set\PYZus{}ylabel}\PY{p}{(}\PY{l+s+s1}{\PYZsq{}}\PY{l+s+s1}{Number of Trip}\PY{l+s+s1}{\PYZsq{}}\PY{p}{)}
\end{Verbatim}


\begin{Verbatim}[commandchars=\\\{\}]
{\color{outcolor}Out[{\color{outcolor}22}]:} <matplotlib.text.Text at 0xc08ad68>
\end{Verbatim}
            
    \begin{center}
    \adjustimage{max size={0.9\linewidth}{0.9\paperheight}}{output_52_1.png}
    \end{center}
    { \hspace*{\fill} \\}
    
    \begin{Verbatim}[commandchars=\\\{\}]
{\color{incolor}In [{\color{incolor}23}]:} \PY{c+c1}{\PYZsh{} rode este comando abaixo caso esteja em dúvida quanto ao resultado esperado}
         \PY{c+c1}{\PYZsh{}usage\PYZus{}plot(trip\PYZus{}data, \PYZsq{}duration\PYZsq{})}
\end{Verbatim}


    Parece muito estranho, não é? Dê uma olhada nos valores de duração no
eixo x. A maioria dos passeios deve ser de 30 minutos ou menos, uma vez
que há taxas de excesso de tempo extra em uma única viagem. A primeira
barra abrange durações de até 1000 minutos, ou mais de 16 horas. Com
base nas estatísticas que obtivemos do \texttt{use\_stats()}, deveríamos
ter esperado algumas viagens com durações muito longas que levem a média
a ser muito superior à mediana: o gráfico mostra isso de forma
dramática, mas inútil.

Ao explorar os dados, muitas vezes você precisará trabalhar com os
parâmetros da função de visualização para facilitar a compreensão dos
dados. É aqui que os filtros vão ajudar você. Comecemos por limitar as
viagens de menos de 60 minutos.

    \begin{Verbatim}[commandchars=\\\{\}]
{\color{incolor}In [{\color{incolor}24}]:} \PY{c+c1}{\PYZsh{} TODO: faça um gráfico de barras para os dados com duração inferior a 60 minutos.}
         \PY{n}{duration\PYZus{}60\PYZus{}min} \PY{o}{=} \PY{n}{trip\PYZus{}data}\PY{p}{[}\PY{l+s+s1}{\PYZsq{}}\PY{l+s+s1}{duration}\PY{l+s+s1}{\PYZsq{}}\PY{p}{]}\PY{p}{[}\PY{n}{trip\PYZus{}data}\PY{p}{[}\PY{l+s+s1}{\PYZsq{}}\PY{l+s+s1}{duration}\PY{l+s+s1}{\PYZsq{}}\PY{p}{]} \PY{o}{\PYZlt{}} \PY{l+m+mi}{60}\PY{p}{]} 
         
         \PY{c+c1}{\PYZsh{}print type(duration\PYZus{}60\PYZus{}min)}
         \PY{c+c1}{\PYZsh{}print duration\PYZus{}60\PYZus{}min.sort\PYZus{}values().tail()}
         
         \PY{n}{duration\PYZus{}60\PYZus{}min\PYZus{}plot} \PY{o}{=} \PY{n}{duration\PYZus{}60\PYZus{}min}\PY{o}{.}\PY{n}{plot}\PY{p}{(}\PY{n}{kind}\PY{o}{=}\PY{l+s+s1}{\PYZsq{}}\PY{l+s+s1}{hist}\PY{l+s+s1}{\PYZsq{}}\PY{p}{,} \PY{n}{color}\PY{o}{=}\PY{l+s+s1}{\PYZsq{}}\PY{l+s+s1}{g}\PY{l+s+s1}{\PYZsq{}}\PY{p}{,} \PY{n}{title}\PY{o}{=}\PY{l+s+s1}{\PYZsq{}}\PY{l+s+s1}{Number of Trips  x  Subscription Type \PYZhy{} Lessed 60 min}\PY{l+s+s1}{\PYZsq{}}\PY{p}{,} \PY{n}{xlim}\PY{o}{=}\PY{p}{[}\PY{l+m+mi}{0}\PY{p}{,}\PY{l+m+mi}{65}\PY{p}{]}\PY{p}{,} \PY{n}{ylim}\PY{o}{=}\PY{p}{[}\PY{l+m+mi}{0}\PY{p}{,}\PY{l+m+mi}{10000}\PY{p}{]}\PY{p}{)}
         
         \PY{n}{duration\PYZus{}60\PYZus{}min\PYZus{}plot}\PY{o}{.}\PY{n}{set\PYZus{}xlabel}\PY{p}{(}\PY{l+s+s1}{\PYZsq{}}\PY{l+s+s1}{Duration [min]}\PY{l+s+s1}{\PYZsq{}}\PY{p}{)}
         \PY{n}{duration\PYZus{}60\PYZus{}min\PYZus{}plot}\PY{o}{.}\PY{n}{set\PYZus{}ylabel}\PY{p}{(}\PY{l+s+s1}{\PYZsq{}}\PY{l+s+s1}{Number of Trip}\PY{l+s+s1}{\PYZsq{}}\PY{p}{)}
\end{Verbatim}


\begin{Verbatim}[commandchars=\\\{\}]
{\color{outcolor}Out[{\color{outcolor}24}]:} <matplotlib.text.Text at 0xc467860>
\end{Verbatim}
            
    \begin{center}
    \adjustimage{max size={0.9\linewidth}{0.9\paperheight}}{output_55_1.png}
    \end{center}
    { \hspace*{\fill} \\}
    
    \begin{Verbatim}[commandchars=\\\{\}]
{\color{incolor}In [{\color{incolor}25}]:} \PY{c+c1}{\PYZsh{} descomente a linha abaixo para verificar o gráfico esperado.}
         \PY{c+c1}{\PYZsh{}usage\PYZus{}plot(trip\PYZus{}data, \PYZsq{}duration\PYZsq{}, [\PYZsq{}duration \PYZlt{} 60\PYZsq{}])}
\end{Verbatim}


    Isso está bem melhor! Você pode ver que a maioria das viagens têm menos
de 30 minutos de duração, mas que você pode fazer mais para melhorar a
apresentação. Uma vez que a duração mínima não é 0, a barra da esquerda
está ligeiramente acima de 0. Nós queremos saber onde existe um limite
perto dos 30 minutos, então ficará mais agradável se tivermos tamanhos
de intervalos (bin sizes) e limites dos intervalos que correspondam a
alguns minutos.

Felizmente, o Pandas e o Matplotlib te dão a opção de resolver ester
problemas. Uma das maneiras de fazê-lo é definindo qual o intervalo no
eixo x (parâmetro range) e quantos intervalos desejamos (bins).

No campo abaixo, faça o ajuste do gráfico para que os limites das barras
se encontrem nas extremidades e que as barras tenham tamanho 5 (0, 5,
10, 15, etc). Se precisar, use a
\href{http://matplotlib.org/api/_as_gen/matplotlib.axes.Axes.hist.html\#matplotlib.axes.Axes.hist}{documentação}.

    \textbf{Tentativa 1}

    \begin{Verbatim}[commandchars=\\\{\}]
{\color{incolor}In [{\color{incolor}26}]:} \PY{c+c1}{\PYZsh{} faça o gráfico ajustado que começará no 0 e terá o bin size de 5}
         
         \PY{n}{duration\PYZus{}60\PYZus{}min\PYZus{}plot} \PY{o}{=} \PY{n}{duration\PYZus{}60\PYZus{}min}\PY{o}{.}\PY{n}{plot}\PY{p}{(}\PY{n}{kind}\PY{o}{=}\PY{l+s+s1}{\PYZsq{}}\PY{l+s+s1}{hist}\PY{l+s+s1}{\PYZsq{}}\PY{p}{,} \PY{n}{color}\PY{o}{=}\PY{l+s+s1}{\PYZsq{}}\PY{l+s+s1}{g}\PY{l+s+s1}{\PYZsq{}}\PY{p}{,} \PY{n}{title}\PY{o}{=}\PY{l+s+s1}{\PYZsq{}}\PY{l+s+s1}{Number of Trips  x  Subscription Type \PYZhy{} Lessed 60 min}\PY{l+s+s1}{\PYZsq{}}\PY{p}{,} \PY{n}{xlim}\PY{o}{=}\PY{p}{[}\PY{l+m+mi}{0}\PY{p}{,}\PY{l+m+mi}{60}\PY{p}{]}\PY{p}{,} \PY{n}{ylim}\PY{o}{=}\PY{p}{[}\PY{l+m+mi}{0}\PY{p}{,}\PY{l+m+mi}{10000}\PY{p}{]}\PY{p}{,} \PY{n+nb}{range}\PY{o}{=}\PY{p}{(}\PY{l+m+mi}{0}\PY{p}{,}\PY{l+m+mi}{60}\PY{p}{)}\PY{p}{,} \PY{n}{bins}\PY{o}{=}\PY{l+m+mi}{12}\PY{p}{,} \PY{n}{rwidth}\PY{o}{=}\PY{l+m+mi}{5}\PY{p}{)}
         
         \PY{n}{duration\PYZus{}60\PYZus{}min\PYZus{}plot}\PY{o}{.}\PY{n}{set\PYZus{}xlabel}\PY{p}{(}\PY{l+s+s1}{\PYZsq{}}\PY{l+s+s1}{Duration [min]}\PY{l+s+s1}{\PYZsq{}}\PY{p}{)}
         \PY{n}{duration\PYZus{}60\PYZus{}min\PYZus{}plot}\PY{o}{.}\PY{n}{set\PYZus{}ylabel}\PY{p}{(}\PY{l+s+s1}{\PYZsq{}}\PY{l+s+s1}{Number of Trip}\PY{l+s+s1}{\PYZsq{}}\PY{p}{)}
\end{Verbatim}


\begin{Verbatim}[commandchars=\\\{\}]
{\color{outcolor}Out[{\color{outcolor}26}]:} <matplotlib.text.Text at 0xcc7ada0>
\end{Verbatim}
            
    \begin{center}
    \adjustimage{max size={0.9\linewidth}{0.9\paperheight}}{output_59_1.png}
    \end{center}
    { \hspace*{\fill} \\}
    
    \begin{Verbatim}[commandchars=\\\{\}]
{\color{incolor}In [{\color{incolor}27}]:} \PY{c+c1}{\PYZsh{} rode esta linha para verificar como deve ficar o seu gráfico}
         \PY{c+c1}{\PYZsh{}usage\PYZus{}plot(trip\PYZus{}data, \PYZsq{}duration\PYZsq{}, [\PYZsq{}duration \PYZlt{} 60\PYZsq{}], boundary = 0, bin\PYZus{}width = 5)}
\end{Verbatim}


    Pequenos ajustes como este podem ser pequenos mas fazem toda a diferença
na entrega de um trabalho de qualidade e com atenção aos detalhes.

    \section{Pergunta 4}\label{pergunta-4}

Analise o histograma do exercicio anterior e responda:

Qual o intervalo de duração com maior quantidade de viagens?

\textbf{Reposta}: O intervalo de duração com a maior quantidade de
viagens é o de 5 a 10 minutos, com mais de 8000 viagens.

    \section{Pergunta 4.1}\label{pergunta-4.1}

Qual viagem de 5 minutos de duração tem a maior quantidade de viagens?
Aproximadamente quantas viagens foram feitas nesta faixa de tempo?

Dica: Identifique a viagens pela origem e destino, calcule quantas
viagem de 5 minutos de duração foram realizadas para cada origem e
destino. Após isso calcule o total de viagens com 5 minutos de duração.

\textbf{Reposta}: - \emph{RESPONDENDO A PRIMEIRA PERGUNTA:}

A viagem de 5 minutos com a maior quantidade de viagens foi De: San
Francisco Para: San Francisco

\begin{itemize}
\tightlist
\item
  \emph{RESPONDENDO A SEGUNDA PERGUNTA:}
\end{itemize}

O total de viagens feitos dentro do període de 5 minutos foi de 11326.

    \textbf{Determinado a resposta da Pergunta 4.1}

    \begin{Verbatim}[commandchars=\\\{\}]
{\color{incolor}In [{\color{incolor}28}]:} \PY{n}{duration\PYZus{}5\PYZus{}min} \PY{o}{=} \PY{n}{trip\PYZus{}data}\PY{p}{[}\PY{p}{[}\PY{l+s+s1}{\PYZsq{}}\PY{l+s+s1}{duration}\PY{l+s+s1}{\PYZsq{}}\PY{p}{,} \PY{l+s+s1}{\PYZsq{}}\PY{l+s+s1}{start\PYZus{}city}\PY{l+s+s1}{\PYZsq{}}\PY{p}{,} \PY{l+s+s1}{\PYZsq{}}\PY{l+s+s1}{end\PYZus{}city}\PY{l+s+s1}{\PYZsq{}}\PY{p}{]}\PY{p}{]}\PY{p}{[}\PY{n}{trip\PYZus{}data}\PY{p}{[}\PY{l+s+s1}{\PYZsq{}}\PY{l+s+s1}{duration}\PY{l+s+s1}{\PYZsq{}}\PY{p}{]} \PY{o}{\PYZlt{}}\PY{o}{=} \PY{l+m+mi}{5}\PY{p}{]} 
         \PY{c+c1}{\PYZsh{}print type(duration\PYZus{}5\PYZus{}min)}
         \PY{c+c1}{\PYZsh{}print duration\PYZus{}5\PYZus{}min.head(20)}
         \PY{c+c1}{\PYZsh{}print len(duration\PYZus{}5\PYZus{}min.index)}
         \PY{n}{trip\PYZus{}5\PYZus{}min} \PY{o}{=}\PY{p}{\PYZob{}}\PY{p}{\PYZcb{}}
         \PY{k}{for} \PY{p}{(}\PY{n}{start\PYZus{}city}\PY{p}{,} \PY{n}{end\PYZus{}city}\PY{p}{,} \PY{n}{duration}\PY{p}{)} \PY{o+ow}{in} \PY{n+nb}{zip}\PY{p}{(}\PY{n}{duration\PYZus{}5\PYZus{}min}\PY{p}{[}\PY{l+s+s1}{\PYZsq{}}\PY{l+s+s1}{start\PYZus{}city}\PY{l+s+s1}{\PYZsq{}}\PY{p}{]}\PY{p}{,} \PY{n}{duration\PYZus{}5\PYZus{}min}\PY{p}{[}\PY{l+s+s1}{\PYZsq{}}\PY{l+s+s1}{end\PYZus{}city}\PY{l+s+s1}{\PYZsq{}}\PY{p}{]}\PY{p}{,} \PY{n}{duration\PYZus{}5\PYZus{}min}\PY{p}{[}\PY{l+s+s1}{\PYZsq{}}\PY{l+s+s1}{duration}\PY{l+s+s1}{\PYZsq{}}\PY{p}{]}\PY{p}{)}\PY{p}{:}
             \PY{k}{if} \PY{n+nb}{str}\PY{p}{(}\PY{l+s+s1}{\PYZsq{}}\PY{l+s+s1}{De: }\PY{l+s+s1}{\PYZsq{}} \PY{o}{+} \PY{n}{start\PYZus{}city} \PY{o}{+} \PY{l+s+s1}{\PYZsq{}}\PY{l+s+s1}{ Para: }\PY{l+s+s1}{\PYZsq{}} \PY{o}{+} \PY{n}{end\PYZus{}city}\PY{p}{)} \PY{o+ow}{not} \PY{o+ow}{in} \PY{n}{trip\PYZus{}5\PYZus{}min}\PY{p}{:}
                 \PY{n}{trip\PYZus{}5\PYZus{}min}\PY{p}{[}\PY{n+nb}{str}\PY{p}{(}\PY{l+s+s1}{\PYZsq{}}\PY{l+s+s1}{De: }\PY{l+s+s1}{\PYZsq{}} \PY{o}{+} \PY{n}{start\PYZus{}city} \PY{o}{+} \PY{l+s+s1}{\PYZsq{}}\PY{l+s+s1}{ Para: }\PY{l+s+s1}{\PYZsq{}} \PY{o}{+} \PY{n}{end\PYZus{}city}\PY{p}{)}\PY{p}{]} \PY{o}{=} \PY{n+nb}{int}\PY{p}{(}\PY{n}{duration}\PY{p}{)}
             \PY{k}{else}\PY{p}{:}
                 \PY{n}{trip\PYZus{}5\PYZus{}min}\PY{p}{[}\PY{n+nb}{str}\PY{p}{(}\PY{l+s+s1}{\PYZsq{}}\PY{l+s+s1}{De: }\PY{l+s+s1}{\PYZsq{}} \PY{o}{+} \PY{n}{start\PYZus{}city} \PY{o}{+} \PY{l+s+s1}{\PYZsq{}}\PY{l+s+s1}{ Para: }\PY{l+s+s1}{\PYZsq{}} \PY{o}{+} \PY{n}{end\PYZus{}city}\PY{p}{)}\PY{p}{]} \PY{o}{+}\PY{o}{=} \PY{n+nb}{int}\PY{p}{(}\PY{n}{duration}\PY{p}{)}
                 
         \PY{c+c1}{\PYZsh{}print trip\PYZus{}5\PYZus{}min}
         \PY{c+c1}{\PYZsh{}print \PYZsq{}\PYZbs{}n\PYZsq{}}
         \PY{k}{print} \PY{l+s+s1}{\PYZsq{}}\PY{l+s+s1}{RESPONDENDO A PRIMEIRA PERGUNTA:}\PY{l+s+s1}{\PYZsq{}}
         \PY{k}{print} \PY{l+s+s1}{\PYZsq{}}\PY{l+s+s1}{A viagem de 5 minutos com a maior quantidade de viagens foi }\PY{l+s+s1}{\PYZsq{}} \PY{o}{+} \PY{n+nb}{sorted}\PY{p}{(}\PY{n}{trip\PYZus{}5\PYZus{}min}\PY{p}{,} \PY{n}{key}\PY{o}{=}\PY{n}{trip\PYZus{}5\PYZus{}min}\PY{o}{.}\PY{n+nf+fm}{\PYZus{}\PYZus{}getitem\PYZus{}\PYZus{}}\PY{p}{)}\PY{p}{[}\PY{o}{\PYZhy{}}\PY{l+m+mi}{1}\PY{p}{]}
         \PY{k}{print} \PY{l+s+s1}{\PYZsq{}}\PY{l+s+se}{\PYZbs{}n}\PY{l+s+s1}{\PYZsq{}}
         \PY{k}{print} \PY{l+s+s1}{\PYZsq{}}\PY{l+s+s1}{RESPONDENDO A SEGUNDA PERGUNTA:}\PY{l+s+s1}{\PYZsq{}}
         \PY{k}{print} \PY{l+s+s1}{\PYZsq{}}\PY{l+s+s1}{O total de viagens feitos dentro de um període de 5 minutos foi de }\PY{l+s+s1}{\PYZsq{}}\PY{p}{,} \PY{n+nb}{sum}\PY{p}{(}\PY{p}{(}\PY{n}{trip\PYZus{}5\PYZus{}min}\PY{o}{.}\PY{n}{values}\PY{p}{(}\PY{p}{)}\PY{p}{)}\PY{p}{)}\PY{p}{,}\PY{l+s+s1}{\PYZsq{}}\PY{l+s+s1}{.}\PY{l+s+s1}{\PYZsq{}}
\end{Verbatim}


    \begin{Verbatim}[commandchars=\\\{\}]
RESPONDENDO A PRIMEIRA PERGUNTA:
A viagem de 5 minutos com a maior quantidade de viagens foi De: San Francisco Para: San Francisco


RESPONDENDO A SEGUNDA PERGUNTA:
O total de viagens feitos dentro de um període de 5 minutos foi de  11326 .

    \end{Verbatim}

    \subsection{Fazendo suas Próprias
Análises}\label{fazendo-suas-pruxf3prias-anuxe1lises}

Agora que você fez alguma exploração em uma pequena amostra do conjunto
de dados, é hora de avançar e reunir todos os dados em um único arquivo
e ver quais tendências você pode encontrar. O código abaixo usará a
mesma função \texttt{summarise\_data()} para processar dados. Depois de
executar a célula abaixo, você terá processado todos os dados em um
único arquivo de dados. Observe que a função não exibirá qualquer saída
enquanto ele é executado, e isso pode demorar um pouco para ser
concluído, pois você tem muito mais dados do que a amostra com a qual
você trabalhou.

    \begin{Verbatim}[commandchars=\\\{\}]
{\color{incolor}In [{\color{incolor}29}]:} \PY{n}{station\PYZus{}data} \PY{o}{=} \PY{p}{[}\PY{l+s+s1}{\PYZsq{}}\PY{l+s+s1}{201402\PYZus{}station\PYZus{}data.csv}\PY{l+s+s1}{\PYZsq{}}\PY{p}{,}
                         \PY{l+s+s1}{\PYZsq{}}\PY{l+s+s1}{201408\PYZus{}station\PYZus{}data.csv}\PY{l+s+s1}{\PYZsq{}}\PY{p}{,}
                         \PY{l+s+s1}{\PYZsq{}}\PY{l+s+s1}{201508\PYZus{}station\PYZus{}data.csv}\PY{l+s+s1}{\PYZsq{}} \PY{p}{]}
         \PY{n}{trip\PYZus{}in} \PY{o}{=} \PY{p}{[}\PY{l+s+s1}{\PYZsq{}}\PY{l+s+s1}{201402\PYZus{}trip\PYZus{}data.csv}\PY{l+s+s1}{\PYZsq{}}\PY{p}{,}
                    \PY{l+s+s1}{\PYZsq{}}\PY{l+s+s1}{201408\PYZus{}trip\PYZus{}data.csv}\PY{l+s+s1}{\PYZsq{}}\PY{p}{,}
                    \PY{l+s+s1}{\PYZsq{}}\PY{l+s+s1}{201508\PYZus{}trip\PYZus{}data.csv}\PY{l+s+s1}{\PYZsq{}} \PY{p}{]}
         \PY{n}{trip\PYZus{}out} \PY{o}{=} \PY{l+s+s1}{\PYZsq{}}\PY{l+s+s1}{babs\PYZus{}y1\PYZus{}y2\PYZus{}summary.csv}\PY{l+s+s1}{\PYZsq{}}
         
         \PY{c+c1}{\PYZsh{} Esta função irá ler as informações das estações e das viagens}
         \PY{c+c1}{\PYZsh{} e escreverá um arquivo processado com o nome trip\PYZus{}out}
         \PY{n}{summarise\PYZus{}data}\PY{p}{(}\PY{n}{trip\PYZus{}in}\PY{p}{,} \PY{n}{station\PYZus{}data}\PY{p}{,} \PY{n}{trip\PYZus{}out}\PY{p}{)}
\end{Verbatim}


    Já que a função \texttt{summarise\_data()} escreveu um arquivo de saída,
a célula acima não precisa ser rodada novamente mesmo que este notebook
seja fechado e uma nova sessão seja criada. Você pode simplesmente ler
os dados novamente e fazer a exploração deste ponto (não esqueça de
executar a parte das funções abaixo ou no começo do notebook caso esteja
em uma nova sessão)

    \begin{Verbatim}[commandchars=\\\{\}]
{\color{incolor}In [{\color{incolor}30}]:} \PY{c+c1}{\PYZsh{} Importa todas as bibliotecas necessárias}
         \PY{o}{\PYZpc{}}\PY{k}{matplotlib} inline
         \PY{k+kn}{import} \PY{n+nn}{csv}
         \PY{k+kn}{from} \PY{n+nn}{datetime} \PY{k+kn}{import} \PY{n}{datetime}
         \PY{k+kn}{import} \PY{n+nn}{numpy} \PY{k+kn}{as} \PY{n+nn}{np}
         \PY{k+kn}{import} \PY{n+nn}{pandas} \PY{k+kn}{as} \PY{n+nn}{pd}
         
         \PY{k+kn}{from} \PY{n+nn}{babs\PYZus{}datacheck} \PY{k+kn}{import} \PY{n}{question\PYZus{}3}
         \PY{k+kn}{from} \PY{n+nn}{babs\PYZus{}visualizations} \PY{k+kn}{import} \PY{n}{usage\PYZus{}stats}\PY{p}{,} \PY{n}{usage\PYZus{}plot}
         \PY{k+kn}{from} \PY{n+nn}{IPython.display} \PY{k+kn}{import} \PY{n}{display}
\end{Verbatim}


    \begin{Verbatim}[commandchars=\\\{\}]
{\color{incolor}In [{\color{incolor}31}]:} \PY{n}{trip\PYZus{}data} \PY{o}{=} \PY{n}{pd}\PY{o}{.}\PY{n}{read\PYZus{}csv}\PY{p}{(}\PY{l+s+s1}{\PYZsq{}}\PY{l+s+s1}{babs\PYZus{}y1\PYZus{}y2\PYZus{}summary.csv}\PY{l+s+s1}{\PYZsq{}}\PY{p}{)}
         \PY{n}{display}\PY{p}{(}\PY{n}{trip\PYZus{}data}\PY{o}{.}\PY{n}{tail}\PY{p}{(}\PY{l+m+mi}{20}\PY{p}{)}\PY{p}{)}
\end{Verbatim}


    
    \begin{verbatim}
          duration  start_date  start_year  start_month  start_hour  weekday  \
669939   25.600000  01/09/2014        2014            9           8        0   
669940   25.750000  01/09/2014        2014            9           8        0   
669941   21.500000  01/09/2014        2014            9           8        0   
669942   10.500000  01/09/2014        2014            9           8        0   
669943    5.550000  01/09/2014        2014            9           8        0   
669944  115.616667  01/09/2014        2014            9           8        0   
669945    7.500000  01/09/2014        2014            9           8        0   
669946    2.683333  01/09/2014        2014            9           8        0   
669947  289.933333  01/09/2014        2014            9           7        0   
669948  288.283333  01/09/2014        2014            9           7        0   
669949    2.816667  01/09/2014        2014            9           7        0   
669950   94.450000  01/09/2014        2014            9           7        0   
669951    7.350000  01/09/2014        2014            9           6        0   
669952    6.633333  01/09/2014        2014            9           5        0   
669953    4.000000  01/09/2014        2014            9           4        0   
669954   10.316667  01/09/2014        2014            9           4        0   
669955  111.866667  01/09/2014        2014            9           3        0   
669956    8.966667  01/09/2014        2014            9           0        0   
669957    9.466667  01/09/2014        2014            9           0        0   
669958    9.483333  01/09/2014        2014            9           0        0   

           start_city       end_city subscription_type  
669939  San Francisco  San Francisco          Customer  
669940  San Francisco  San Francisco          Customer  
669941  San Francisco  San Francisco        Subscriber  
669942  San Francisco  San Francisco        Subscriber  
669943  San Francisco  San Francisco        Subscriber  
669944  San Francisco  San Francisco          Customer  
669945  San Francisco  San Francisco        Subscriber  
669946  San Francisco  San Francisco        Subscriber  
669947  Mountain View  Mountain View          Customer  
669948  Mountain View  Mountain View          Customer  
669949  San Francisco  San Francisco        Subscriber  
669950       San Jose       San Jose          Customer  
669951  San Francisco  San Francisco        Subscriber  
669952  San Francisco  San Francisco        Subscriber  
669953  San Francisco  San Francisco        Subscriber  
669954  San Francisco  San Francisco        Subscriber  
669955  San Francisco  San Francisco          Customer  
669956  San Francisco  San Francisco          Customer  
669957  San Francisco  San Francisco          Customer  
669958  San Francisco  San Francisco          Customer  
    \end{verbatim}

    
    \begin{Verbatim}[commandchars=\\\{\}]
{\color{incolor}In [{\color{incolor}32}]:} \PY{k}{print} \PY{l+s+s1}{\PYZsq{}}\PY{l+s+s1}{O total de viagens realizadas nesse novo arquivo é de }\PY{l+s+s1}{\PYZsq{}}\PY{p}{,} \PY{n+nb}{len}\PY{p}{(}\PY{n}{trip\PYZus{}data}\PY{o}{.}\PY{n}{index}\PY{p}{)}
         
         \PY{k}{print} \PY{l+s+s2}{\PYZdq{}}\PY{l+s+s2}{A }\PY{l+s+s2}{\PYZsq{}}\PY{l+s+s2}{start\PYZus{}hour}\PY{l+s+s2}{\PYZsq{}}\PY{l+s+s2}{ máxima é }\PY{l+s+s2}{\PYZdq{}}\PY{p}{,} \PY{n+nb}{max}\PY{p}{(}\PY{n}{trip\PYZus{}data}\PY{p}{[}\PY{l+s+s1}{\PYZsq{}}\PY{l+s+s1}{start\PYZus{}hour}\PY{l+s+s1}{\PYZsq{}}\PY{p}{]}\PY{o}{.}\PY{n}{values}\PY{p}{)}\PY{p}{,} \PY{l+s+s2}{\PYZdq{}}\PY{l+s+s2}{ e a mínima é de }\PY{l+s+s2}{\PYZdq{}}\PY{p}{,} \PY{n+nb}{min}\PY{p}{(}\PY{n}{trip\PYZus{}data}\PY{p}{[}\PY{l+s+s1}{\PYZsq{}}\PY{l+s+s1}{start\PYZus{}hour}\PY{l+s+s1}{\PYZsq{}}\PY{p}{]}\PY{o}{.}\PY{n}{values}\PY{p}{)}
\end{Verbatim}


    \begin{Verbatim}[commandchars=\\\{\}]
O total de viagens realizadas nesse novo arquivo é de  669959
A 'start\_hour' máxima é  23  e a mínima é de  0

    \end{Verbatim}

    \paragraph{Agora é a SUA vez de fazer a exploração do dataset (do
conjunto de dados)
completo.}\label{agora-uxe9-a-sua-vez-de-fazer-a-explorauxe7uxe3o-do-dataset-do-conjunto-de-dados-completo.}

Aproveite para fazer filtros nos dados e tentar encontrar padrões nos
dados.

    Explore algumas variáveis diferentes usando o mesmo racional acima e
tome nota de algumas tendências que você encontra. Sinta-se livre para
criar células adicionais se quiser explorar o conjunto de dados de
outras maneiras ou de várias maneiras.

\begin{quote}
\textbf{Dica}: para adicionar células adicionais a um notebook, você
pode usar as opções "Inserir célula acima" (Insert Cell Above) e "Insert
Cell Below" na barra de menu acima. Há também um ícone na barra de
ferramentas para adicionar novas células, com ícones adicionais para
mover as células para cima e para baixo do documento. Por padrão, as
novas células são do tipo de código; Você também pode especificar o tipo
de célula (por exemplo, Código ou Markdown) das células selecionadas no
menu Cell ou no menu dropdown na barra de ferramentas.
\end{quote}

Um feito com suas explorações, copie as duas visualizações que você
achou mais interessantes nas células abaixo e responda as seguintes
perguntas com algumas frases descrevendo o que você encontrou e por que
você selecionou os números. Certifique-se de que você ajusta o número de
caixas ou os limites da bandeja para que efetivamente transmitam os
resultados dos dados. Sinta-se livre para complementar isso com
quaisquer números adicionais gerados a partir de \texttt{use\_stats()}
ou coloque visualizações múltiplas para suportar suas observações.

    Para ver alguns outros tipos de gráficos que o matplotlib (padrão do
Pandas) possui, leia
\href{https://www.labri.fr/perso/nrougier/teaching/matplotlib/\#other-types-of-plots}{este
artigo}.

Para entender um pouco mais como e quais gráficos podem ser úteis, leia
\href{https://www.tableau.com/sites/default/files/media/Whitepapers/which_chart_v6_ptb.pdf}{este
documento}. Ele lhe dará um pouco de idéia de como mostrar os dados de
forma mais acertada

    \textbf{OBJEIVO DA 'ANÁLISE PRÓPRIA': } O objetivo dessa primeira
análise é entender quais são os horários de maior movimento comparando
os dois tipo de usuarios (Customer vs Subscriber). Com esses horários
determinados, é possivel analisar dados estatisticos sobre a 'duration'
nesses horários e ainda analisar a diferença desses dados para os dias
da semana e finais de semana, onde monday=0, ..., sunday=6.

    \begin{Verbatim}[commandchars=\\\{\}]
{\color{incolor}In [{\color{incolor}33}]:} \PY{c+c1}{\PYZsh{}Separando por subscription\PYZus{}type}
         
         
         \PY{n}{customer} \PY{o}{=} \PY{n}{trip\PYZus{}data}\PY{p}{[}\PY{p}{[}\PY{l+s+s1}{\PYZsq{}}\PY{l+s+s1}{duration}\PY{l+s+s1}{\PYZsq{}}\PY{p}{,} \PY{l+s+s1}{\PYZsq{}}\PY{l+s+s1}{start\PYZus{}hour}\PY{l+s+s1}{\PYZsq{}}\PY{p}{,} \PY{l+s+s1}{\PYZsq{}}\PY{l+s+s1}{weekday}\PY{l+s+s1}{\PYZsq{}}\PY{p}{,} \PY{l+s+s1}{\PYZsq{}}\PY{l+s+s1}{subscription\PYZus{}type}\PY{l+s+s1}{\PYZsq{}}\PY{p}{]}\PY{p}{]}\PY{p}{[}\PY{n}{trip\PYZus{}data}\PY{p}{[}\PY{l+s+s1}{\PYZsq{}}\PY{l+s+s1}{subscription\PYZus{}type}\PY{l+s+s1}{\PYZsq{}}\PY{p}{]} \PY{o}{==} \PY{l+s+s1}{\PYZsq{}}\PY{l+s+s1}{Customer}\PY{l+s+s1}{\PYZsq{}}\PY{p}{]}
         \PY{n}{subscriber} \PY{o}{=} \PY{n}{trip\PYZus{}data}\PY{p}{[}\PY{p}{[}\PY{l+s+s1}{\PYZsq{}}\PY{l+s+s1}{duration}\PY{l+s+s1}{\PYZsq{}}\PY{p}{,} \PY{l+s+s1}{\PYZsq{}}\PY{l+s+s1}{start\PYZus{}hour}\PY{l+s+s1}{\PYZsq{}}\PY{p}{,} \PY{l+s+s1}{\PYZsq{}}\PY{l+s+s1}{weekday}\PY{l+s+s1}{\PYZsq{}}\PY{p}{,} \PY{l+s+s1}{\PYZsq{}}\PY{l+s+s1}{subscription\PYZus{}type}\PY{l+s+s1}{\PYZsq{}}\PY{p}{]}\PY{p}{]}\PY{p}{[}\PY{n}{trip\PYZus{}data}\PY{p}{[}\PY{l+s+s1}{\PYZsq{}}\PY{l+s+s1}{subscription\PYZus{}type}\PY{l+s+s1}{\PYZsq{}}\PY{p}{]} \PY{o}{==} \PY{l+s+s1}{\PYZsq{}}\PY{l+s+s1}{Subscriber}\PY{l+s+s1}{\PYZsq{}}\PY{p}{]}
         
         \PY{k}{print} \PY{l+s+s2}{\PYZdq{}}\PY{l+s+s2}{The type of variable }\PY{l+s+s2}{\PYZsq{}}\PY{l+s+s2}{customer}\PY{l+s+s2}{\PYZsq{}}\PY{l+s+s2}{ is }\PY{l+s+s2}{\PYZdq{}}\PY{p}{,} \PY{n+nb}{type}\PY{p}{(}\PY{n}{customer}\PY{p}{)}
         \PY{k}{print} \PY{l+s+s2}{\PYZdq{}}\PY{l+s+s2}{The type of variable }\PY{l+s+s2}{\PYZsq{}}\PY{l+s+s2}{subscriber}\PY{l+s+s2}{\PYZsq{}}\PY{l+s+s2}{ is }\PY{l+s+s2}{\PYZdq{}}\PY{p}{,} \PY{n+nb}{type}\PY{p}{(}\PY{n}{subscriber}\PY{p}{)}
         \PY{k}{print} \PY{l+s+s1}{\PYZsq{}}\PY{l+s+se}{\PYZbs{}n}\PY{l+s+s1}{\PYZsq{}}
         
         \PY{k}{print} \PY{n}{customer}\PY{o}{.}\PY{n}{head}\PY{p}{(}\PY{p}{)}
         \PY{k}{print} \PY{n}{subscriber}\PY{o}{.}\PY{n}{head}\PY{p}{(}\PY{p}{)}
         \PY{k}{print} \PY{l+s+s1}{\PYZsq{}}\PY{l+s+se}{\PYZbs{}n}\PY{l+s+s1}{\PYZsq{}}
         
         \PY{c+c1}{\PYZsh{}print customer.tail()}
         \PY{c+c1}{\PYZsh{}print subscriber.tail()}
         \PY{c+c1}{\PYZsh{}print \PYZsq{}\PYZbs{}n\PYZsq{}}
         
         
         \PY{c+c1}{\PYZsh{}print len(list(subscriber.values)) }
         \PY{c+c1}{\PYZsh{}print len(list(customer.values))}
         \PY{c+c1}{\PYZsh{}print \PYZsq{}\PYZbs{}n\PYZsq{}}
         \PY{k}{print} \PY{l+s+s1}{\PYZsq{}}\PY{l+s+s1}{O total de viagens realizadas é de }\PY{l+s+s1}{\PYZsq{}}\PY{p}{,} \PY{n+nb}{len}\PY{p}{(}\PY{n}{subscriber}\PY{o}{.}\PY{n}{index}\PY{p}{)} \PY{o}{+} \PY{n+nb}{len}\PY{p}{(}\PY{n}{customer}\PY{o}{.}\PY{n}{index}\PY{p}{)}\PY{p}{,} \PY{l+s+s1}{\PYZsq{}}\PY{l+s+s1}{. Condizente com o valor anteriormente encontrado.}\PY{l+s+s1}{\PYZsq{}}
\end{Verbatim}


    \begin{Verbatim}[commandchars=\\\{\}]
The type of variable 'customer' is  <class 'pandas.core.frame.DataFrame'>
The type of variable 'subscriber' is  <class 'pandas.core.frame.DataFrame'>


    duration  start\_hour  weekday subscription\_type
24  2.683333          10        3          Customer
30  2.883333          11        3          Customer
41  3.133333          19        3          Customer
48  3.483333          17        3          Customer
51  3.566667          17        3          Customer
   duration  start\_hour  weekday subscription\_type
0  1.050000          14        3        Subscriber
1  1.166667          14        3        Subscriber
2  1.183333          10        3        Subscriber
3  1.283333          11        3        Subscriber
4  1.383333          12        3        Subscriber


O total de viagens realizadas é de  669959 . Condizente com o valor anteriormente encontrado.

    \end{Verbatim}

    \begin{Verbatim}[commandchars=\\\{\}]
{\color{incolor}In [{\color{incolor}34}]:} \PY{c+c1}{\PYZsh{}Definindo a função que retorna o start\PYZus{}hour de maior movimento}
         
         \PY{k}{def} \PY{n+nf}{busy\PYZus{}hour}\PY{p}{(}\PY{n}{data\PYZus{}file}\PY{p}{)}\PY{p}{:}
             \PY{c+c1}{\PYZsh{}empty dict with all possibles start\PYZus{}hour}
             \PY{n}{start\PYZus{}hours} \PY{o}{=} \PY{p}{\PYZob{}}\PY{l+m+mi}{0}\PY{p}{:}\PY{n+nb}{int}\PY{p}{(}\PY{p}{)}\PY{p}{,} \PY{l+m+mi}{1}\PY{p}{:}\PY{n+nb}{int}\PY{p}{(}\PY{p}{)}\PY{p}{,} \PY{l+m+mi}{2}\PY{p}{:}\PY{n+nb}{int}\PY{p}{(}\PY{p}{)}\PY{p}{,} \PY{l+m+mi}{3}\PY{p}{:}\PY{n+nb}{int}\PY{p}{(}\PY{p}{)}\PY{p}{,} \PY{l+m+mi}{4}\PY{p}{:}\PY{n+nb}{int}\PY{p}{(}\PY{p}{)}\PY{p}{,} \PY{l+m+mi}{5}\PY{p}{:}\PY{n+nb}{int}\PY{p}{(}\PY{p}{)}\PY{p}{,} \PY{l+m+mi}{6}\PY{p}{:}\PY{n+nb}{int}\PY{p}{(}\PY{p}{)}\PY{p}{,} \PY{l+m+mi}{7}\PY{p}{:}\PY{n+nb}{int}\PY{p}{(}\PY{p}{)}\PY{p}{,} \PY{l+m+mi}{8}\PY{p}{:}\PY{n+nb}{int}\PY{p}{(}\PY{p}{)}\PY{p}{,} \PY{l+m+mi}{9}\PY{p}{:}\PY{n+nb}{int}\PY{p}{(}\PY{p}{)}\PY{p}{,} \PY{l+m+mi}{10}\PY{p}{:}\PY{n+nb}{int}\PY{p}{(}\PY{p}{)}\PY{p}{,} \PY{l+m+mi}{11}\PY{p}{:}\PY{n+nb}{int}\PY{p}{(}\PY{p}{)}\PY{p}{,} \PY{l+m+mi}{12}\PY{p}{:}\PY{n+nb}{int}\PY{p}{(}\PY{p}{)}\PY{p}{,}
                            \PY{l+m+mi}{13}\PY{p}{:}\PY{n+nb}{int}\PY{p}{(}\PY{p}{)}\PY{p}{,} \PY{l+m+mi}{14}\PY{p}{:}\PY{n+nb}{int}\PY{p}{(}\PY{p}{)}\PY{p}{,} \PY{l+m+mi}{15}\PY{p}{:}\PY{n+nb}{int}\PY{p}{(}\PY{p}{)}\PY{p}{,} \PY{l+m+mi}{16}\PY{p}{:}\PY{n+nb}{int}\PY{p}{(}\PY{p}{)}\PY{p}{,} \PY{l+m+mi}{17}\PY{p}{:}\PY{n+nb}{int}\PY{p}{(}\PY{p}{)}\PY{p}{,} \PY{l+m+mi}{18}\PY{p}{:}\PY{n+nb}{int}\PY{p}{(}\PY{p}{)}\PY{p}{,} \PY{l+m+mi}{19}\PY{p}{:}\PY{n+nb}{int}\PY{p}{(}\PY{p}{)}\PY{p}{,} \PY{l+m+mi}{20}\PY{p}{:}\PY{n+nb}{int}\PY{p}{(}\PY{p}{)}\PY{p}{,} \PY{l+m+mi}{21}\PY{p}{:}\PY{n+nb}{int}\PY{p}{(}\PY{p}{)}\PY{p}{,} \PY{l+m+mi}{22}\PY{p}{:}\PY{n+nb}{int}\PY{p}{(}\PY{p}{)}\PY{p}{,} \PY{l+m+mi}{23}\PY{p}{:}\PY{n+nb}{int}\PY{p}{(}\PY{p}{)}\PY{p}{\PYZcb{}}
             \PY{c+c1}{\PYZsh{}print type(start\PYZus{}hours.keys()[0])}
             \PY{k}{for} \PY{n}{hour} \PY{o+ow}{in} \PY{n}{data\PYZus{}file}\PY{p}{[}\PY{l+s+s1}{\PYZsq{}}\PY{l+s+s1}{start\PYZus{}hour}\PY{l+s+s1}{\PYZsq{}}\PY{p}{]}\PY{p}{:}
                 \PY{n}{start\PYZus{}hours}\PY{p}{[}\PY{n}{hour}\PY{p}{]} \PY{o}{+}\PY{o}{=} \PY{l+m+mi}{1}
             \PY{k}{return} \PY{n}{start\PYZus{}hours}
\end{Verbatim}


    \begin{Verbatim}[commandchars=\\\{\}]
{\color{incolor}In [{\color{incolor}35}]:} \PY{c+c1}{\PYZsh{}aplicando para os dois tipos de usuario}
         \PY{n}{hours\PYZus{}subscriber} \PY{o}{=} \PY{n}{busy\PYZus{}hour}\PY{p}{(}\PY{n}{subscriber}\PY{p}{)}
         \PY{n}{hours\PYZus{}customer} \PY{o}{=} \PY{n}{busy\PYZus{}hour}\PY{p}{(}\PY{n}{customer}\PY{p}{)}
\end{Verbatim}


    \begin{Verbatim}[commandchars=\\\{\}]
{\color{incolor}In [{\color{incolor}36}]:} \PY{c+c1}{\PYZsh{}Usando matplot.pylab para plotar o comportamento e identificar os horários de maior movimento}
         \PY{k+kn}{import} \PY{n+nn}{matplotlib.pyplot} \PY{k+kn}{as} \PY{n+nn}{plt}
\end{Verbatim}


    \textbf{Maior movimento para 'subscriber':}

    \begin{Verbatim}[commandchars=\\\{\}]
{\color{incolor}In [{\color{incolor}37}]:} \PY{n}{subscriber\PYZus{}x} \PY{o}{=} \PY{n}{hours\PYZus{}subscriber}\PY{o}{.}\PY{n}{keys}\PY{p}{(}\PY{p}{)}
         \PY{n}{subscriber\PYZus{}y} \PY{o}{=} \PY{n}{hours\PYZus{}subscriber}\PY{o}{.}\PY{n}{values}\PY{p}{(}\PY{p}{)}
         
         \PY{k}{print} \PY{l+s+s2}{\PYZdq{}}\PY{l+s+s2}{O total de viagens ralizadas por usuários do tipo }\PY{l+s+s2}{\PYZsq{}}\PY{l+s+s2}{subscriber}\PY{l+s+s2}{\PYZsq{}}\PY{l+s+s2}{ é de }\PY{l+s+s2}{\PYZdq{}}\PY{p}{,} \PY{n+nb}{sum}\PY{p}{(}\PY{n}{subscriber\PYZus{}y}\PY{p}{)}
         
         \PY{n}{plt}\PY{o}{.}\PY{n}{plot}\PY{p}{(}\PY{n}{subscriber\PYZus{}x}\PY{p}{,} \PY{n}{subscriber\PYZus{}y}\PY{p}{)}
         \PY{n}{plt}\PY{o}{.}\PY{n}{xlabel}\PY{p}{(}\PY{l+s+s1}{\PYZsq{}}\PY{l+s+s1}{Start Hour}\PY{l+s+s1}{\PYZsq{}}\PY{p}{)}
         \PY{n}{plt}\PY{o}{.}\PY{n}{ylabel}\PY{p}{(}\PY{l+s+s1}{\PYZsq{}}\PY{l+s+s1}{Frequency}\PY{l+s+s1}{\PYZsq{}}\PY{p}{)}
         \PY{n}{plt}\PY{o}{.}\PY{n}{title}\PY{p}{(}\PY{l+s+s1}{\PYZsq{}}\PY{l+s+s1}{Busy Hours for Subscriber}\PY{l+s+s1}{\PYZsq{}}\PY{p}{)}
         \PY{n}{plt}\PY{o}{.}\PY{n}{axis}\PY{p}{(}\PY{p}{[}\PY{l+m+mi}{0}\PY{p}{,} \PY{l+m+mi}{23}\PY{p}{,} \PY{l+m+mi}{0}\PY{p}{,} \PY{l+m+mi}{83000}\PY{p}{]}\PY{p}{)}
\end{Verbatim}


    \begin{Verbatim}[commandchars=\\\{\}]
O total de viagens ralizadas por usuários do tipo 'subscriber' é de  566746

    \end{Verbatim}

\begin{Verbatim}[commandchars=\\\{\}]
{\color{outcolor}Out[{\color{outcolor}37}]:} [0, 23, 0, 83000]
\end{Verbatim}
            
    \begin{center}
    \adjustimage{max size={0.9\linewidth}{0.9\paperheight}}{output_81_2.png}
    \end{center}
    { \hspace*{\fill} \\}
    
    \emph{Ánalise do Gráfico 'Busy Hours for Subscriber':}

\begin{verbatim}
Podemos perceber que existem dois horários de grandes picos de movimento no caso do usuario 'subscriber'. O primeiro é no período entre as 7 e 9h e o segundo das entre as 16 e as 18h. Também há um pico menor aproximadamente as 12h.
\end{verbatim}

    \textbf{Maior movimento para 'customer':}

    \begin{Verbatim}[commandchars=\\\{\}]
{\color{incolor}In [{\color{incolor}38}]:} \PY{n}{customer\PYZus{}x} \PY{o}{=} \PY{n}{hours\PYZus{}customer}\PY{o}{.}\PY{n}{keys}\PY{p}{(}\PY{p}{)}
         \PY{n}{customer\PYZus{}y} \PY{o}{=} \PY{n}{hours\PYZus{}customer}\PY{o}{.}\PY{n}{values}\PY{p}{(}\PY{p}{)}
         
         \PY{k}{print}  \PY{l+s+s2}{\PYZdq{}}\PY{l+s+s2}{O total de viagens ralizadas por usuários do tipo }\PY{l+s+s2}{\PYZsq{}}\PY{l+s+s2}{customer}\PY{l+s+s2}{\PYZsq{}}\PY{l+s+s2}{ é de }\PY{l+s+s2}{\PYZdq{}}\PY{p}{,} \PY{n+nb}{sum}\PY{p}{(}\PY{n}{customer\PYZus{}y}\PY{p}{)}
         
         \PY{n}{plt}\PY{o}{.}\PY{n}{plot}\PY{p}{(}\PY{n}{customer\PYZus{}x}\PY{p}{,} \PY{n}{customer\PYZus{}y}\PY{p}{)}
         \PY{n}{plt}\PY{o}{.}\PY{n}{xlabel}\PY{p}{(}\PY{l+s+s1}{\PYZsq{}}\PY{l+s+s1}{Start Hour}\PY{l+s+s1}{\PYZsq{}}\PY{p}{)}
         \PY{n}{plt}\PY{o}{.}\PY{n}{ylabel}\PY{p}{(}\PY{l+s+s1}{\PYZsq{}}\PY{l+s+s1}{Frequency}\PY{l+s+s1}{\PYZsq{}}\PY{p}{)}
         \PY{n}{plt}\PY{o}{.}\PY{n}{title}\PY{p}{(}\PY{l+s+s1}{\PYZsq{}}\PY{l+s+s1}{Busy Hours for Customer}\PY{l+s+s1}{\PYZsq{}}\PY{p}{)}
         \PY{n}{plt}\PY{o}{.}\PY{n}{axis}\PY{p}{(}\PY{p}{[}\PY{l+m+mi}{0}\PY{p}{,} \PY{l+m+mi}{23}\PY{p}{,} \PY{l+m+mi}{0}\PY{p}{,} \PY{l+m+mi}{10000}\PY{p}{]}\PY{p}{)}
\end{Verbatim}


    \begin{Verbatim}[commandchars=\\\{\}]
O total de viagens ralizadas por usuários do tipo 'customer' é de  103213

    \end{Verbatim}

\begin{Verbatim}[commandchars=\\\{\}]
{\color{outcolor}Out[{\color{outcolor}38}]:} [0, 23, 0, 10000]
\end{Verbatim}
            
    \begin{center}
    \adjustimage{max size={0.9\linewidth}{0.9\paperheight}}{output_84_2.png}
    \end{center}
    { \hspace*{\fill} \\}
    
    \emph{Ánalise do Gráfico 'Busy Hours for Subscriber':}

\begin{verbatim}
Podemos perceber que existem um único horário de grande movimento entre 12 e 17h.
\end{verbatim}

    \begin{Verbatim}[commandchars=\\\{\}]
{\color{incolor}In [{\color{incolor}39}]:} \PY{k}{print} \PY{l+s+s2}{\PYZdq{}}\PY{l+s+s2}{Viagens dos usuários do tipo }\PY{l+s+s2}{\PYZsq{}}\PY{l+s+s2}{subscriber}\PY{l+s+s2}{\PYZsq{}}\PY{l+s+s2}{ representam }\PY{l+s+s2}{\PYZdq{}}\PY{p}{,} \PY{p}{(}\PY{n+nb}{float}\PY{p}{(}\PY{n+nb}{sum}\PY{p}{(}\PY{n}{subscriber\PYZus{}y}\PY{p}{)}\PY{p}{)}\PY{o}{/}\PY{n+nb}{float}\PY{p}{(}\PY{n+nb}{len}\PY{p}{(}\PY{n}{trip\PYZus{}data}\PY{o}{.}\PY{n}{index}\PY{p}{)}\PY{p}{)}\PY{p}{)}\PY{o}{*}\PY{l+m+mi}{100}\PY{p}{,}\PY{l+s+s2}{\PYZdq{}}\PY{l+s+si}{\PYZpc{} d}\PY{l+s+s2}{o total de viagens.}\PY{l+s+s2}{\PYZdq{}} 
         
         \PY{k}{print} \PY{l+s+s2}{\PYZdq{}}\PY{l+s+s2}{Já os usuários do tipo }\PY{l+s+s2}{\PYZsq{}}\PY{l+s+s2}{customer}\PY{l+s+s2}{\PYZsq{}}\PY{l+s+s2}{ representam }\PY{l+s+s2}{\PYZdq{}}\PY{p}{,} \PY{p}{(}\PY{n+nb}{float}\PY{p}{(}\PY{n+nb}{sum}\PY{p}{(}\PY{n}{customer\PYZus{}y}\PY{p}{)}\PY{p}{)}\PY{o}{/}\PY{n+nb}{float}\PY{p}{(}\PY{n+nb}{len}\PY{p}{(}\PY{n}{trip\PYZus{}data}\PY{o}{.}\PY{n}{index}\PY{p}{)}\PY{p}{)}\PY{p}{)}\PY{o}{*}\PY{l+m+mi}{100}\PY{p}{,}\PY{l+s+s2}{\PYZdq{}}\PY{l+s+s2}{\PYZpc{}}\PY{l+s+s2}{.}\PY{l+s+s2}{\PYZdq{}}
\end{Verbatim}


    \begin{Verbatim}[commandchars=\\\{\}]
Viagens dos usuários do tipo 'subscriber' representam  84.5941318797 \% do total de viagens.
Já os usuários do tipo 'customer' representam  15.4058681203 \%.

    \end{Verbatim}

    É possível observar que os diferentes tipos de usuários apresentam
diferentes comportamento no que diz a respeito dos horários de
urlização. Agora, a análise seguirá para análisar informações
estatísticas sobre a duração de seus viagens e dias de utlização,
comparando a utilização em dias de semana e fianis de semana.

    \subsection{Análise estatística da duração das viagens nos horários de
maior
movmento}\label{anuxe1lise-estatuxedstica-da-durauxe7uxe3o-das-viagens-nos-horuxe1rios-de-maior-movmento}

    \subsubsection{Para usuários do tipo
'subscriber':}\label{para-usuuxe1rios-do-tipo-subscriber}

A análise será feita no intervalo de maior movimento, ou seja, entre as
6 e as 9h. Intervalo definido com base no gráfico 'Busy Hours for
Subscriber'.

    \begin{Verbatim}[commandchars=\\\{\}]
{\color{incolor}In [{\color{incolor}40}]:} \PY{n}{subscriber\PYZus{}busy} \PY{o}{=} \PY{n}{subscriber}\PY{p}{[}\PY{p}{[}\PY{l+s+s1}{\PYZsq{}}\PY{l+s+s1}{duration}\PY{l+s+s1}{\PYZsq{}}\PY{p}{,} \PY{l+s+s1}{\PYZsq{}}\PY{l+s+s1}{weekday}\PY{l+s+s1}{\PYZsq{}}\PY{p}{,} \PY{l+s+s1}{\PYZsq{}}\PY{l+s+s1}{start\PYZus{}hour}\PY{l+s+s1}{\PYZsq{}}\PY{p}{]}\PY{p}{]}\PY{p}{[}\PY{p}{(}\PY{n}{subscriber}\PY{p}{[}\PY{l+s+s1}{\PYZsq{}}\PY{l+s+s1}{start\PYZus{}hour}\PY{l+s+s1}{\PYZsq{}}\PY{p}{]}\PY{o}{\PYZgt{}}\PY{o}{=}\PY{l+m+mi}{7}\PY{p}{)} \PY{o}{\PYZam{}} \PY{p}{(}\PY{n}{subscriber}\PY{p}{[}\PY{l+s+s1}{\PYZsq{}}\PY{l+s+s1}{start\PYZus{}hour}\PY{l+s+s1}{\PYZsq{}}\PY{p}{]}\PY{o}{\PYZlt{}}\PY{o}{=}\PY{l+m+mi}{9}\PY{p}{)}\PY{p}{]}
         
         \PY{c+c1}{\PYZsh{}testar funcionamento}
         \PY{k}{print} \PY{l+s+s1}{\PYZsq{}}\PY{l+s+s1}{The maximum start\PYZus{}hour is }\PY{l+s+s1}{\PYZsq{}}\PY{p}{,} \PY{n+nb}{max}\PY{p}{(}\PY{n}{subscriber\PYZus{}busy}\PY{p}{[}\PY{l+s+s1}{\PYZsq{}}\PY{l+s+s1}{start\PYZus{}hour}\PY{l+s+s1}{\PYZsq{}}\PY{p}{]}\PY{o}{.}\PY{n}{values}\PY{p}{)}
         \PY{k}{print} \PY{l+s+s1}{\PYZsq{}}\PY{l+s+s1}{The minimum start\PYZus{}hour is }\PY{l+s+s1}{\PYZsq{}}\PY{p}{,} \PY{n+nb}{min}\PY{p}{(}\PY{n}{subscriber\PYZus{}busy}\PY{p}{[}\PY{l+s+s1}{\PYZsq{}}\PY{l+s+s1}{start\PYZus{}hour}\PY{l+s+s1}{\PYZsq{}}\PY{p}{]}\PY{o}{.}\PY{n}{values}\PY{p}{)}
         \PY{k}{print} \PY{n+nb}{type}\PY{p}{(}\PY{n}{subscriber\PYZus{}busy}\PY{p}{)}
\end{Verbatim}


    \begin{Verbatim}[commandchars=\\\{\}]
The maximum start\_hour is  9
The minimum start\_hour is  7
<class 'pandas.core.frame.DataFrame'>

    \end{Verbatim}

    \begin{Verbatim}[commandchars=\\\{\}]
{\color{incolor}In [{\color{incolor}41}]:} \PY{c+c1}{\PYZsh{}Estatistica com describe() function}
         \PY{k}{print} \PY{n}{subscriber\PYZus{}busy}\PY{p}{[}\PY{l+s+s1}{\PYZsq{}}\PY{l+s+s1}{duration}\PY{l+s+s1}{\PYZsq{}}\PY{p}{]}\PY{o}{.}\PY{n}{describe}\PY{p}{(}\PY{p}{)}
\end{Verbatim}


    \begin{Verbatim}[commandchars=\\\{\}]
count    182131.000000
mean          9.517797
std          36.072745
min           1.000000
25\%           5.616667
50\%           8.083333
75\%          11.116667
max       10322.033333
Name: duration, dtype: float64

    \end{Verbatim}

    \emph{Análise sobre os dados estatísticos:}

Os dados mostram um comportamento de trajetos com tempos de durações
mais curtos, onde não há taxamento extra. Os valores a cima do terceiro
quartil não é representativo, visto que 75\% ocorre em aproximadamente
11 minutos e o desvio padrão é alto, bem como o há valores muito altos
que devem estar mascarando o padrão desse tipo de usuário. Portanto,
podemos resolver isso retirando os dados que estão a cima dos 75\% dos
dados.

    \begin{Verbatim}[commandchars=\\\{\}]
{\color{incolor}In [{\color{incolor}42}]:} \PY{c+c1}{\PYZsh{}Removendo dados a cima de 75\PYZpc{} dos dados}
         \PY{n}{subscriber\PYZus{}busy\PYZus{}clean} \PY{o}{=} \PY{n}{subscriber\PYZus{}busy}\PY{p}{[}\PY{p}{[}\PY{l+s+s1}{\PYZsq{}}\PY{l+s+s1}{duration}\PY{l+s+s1}{\PYZsq{}}\PY{p}{]}\PY{p}{]}\PY{p}{[}\PY{n}{subscriber\PYZus{}busy}\PY{p}{[}\PY{l+s+s1}{\PYZsq{}}\PY{l+s+s1}{duration}\PY{l+s+s1}{\PYZsq{}}\PY{p}{]} \PY{o}{\PYZlt{}}\PY{o}{=} \PY{n}{subscriber\PYZus{}busy}\PY{p}{[}\PY{l+s+s1}{\PYZsq{}}\PY{l+s+s1}{duration}\PY{l+s+s1}{\PYZsq{}}\PY{p}{]}\PY{o}{.}\PY{n}{quantile}\PY{p}{(}\PY{o}{.}\PY{l+m+mi}{75}\PY{p}{)}\PY{p}{]}
         \PY{c+c1}{\PYZsh{}Testa para analisar se a limpeza funcionou}
         \PY{n}{subscriber\PYZus{}busy\PYZus{}clean}\PY{o}{.}\PY{n}{sort\PYZus{}values}\PY{p}{(}\PY{l+s+s1}{\PYZsq{}}\PY{l+s+s1}{duration}\PY{l+s+s1}{\PYZsq{}}\PY{p}{)}\PY{o}{.}\PY{n}{tail}\PY{p}{(}\PY{p}{)}
\end{Verbatim}


\begin{Verbatim}[commandchars=\\\{\}]
{\color{outcolor}Out[{\color{outcolor}42}]:}          duration
         71925   11.116667
         416194  11.116667
         638018  11.116667
         574574  11.116667
         397300  11.116667
\end{Verbatim}
            
    \begin{Verbatim}[commandchars=\\\{\}]
{\color{incolor}In [{\color{incolor}43}]:} \PY{c+c1}{\PYZsh{}Reexecutando funcao describe() para os dados limpos}
         \PY{n}{subscriber\PYZus{}busy\PYZus{}clean}\PY{o}{.}\PY{n}{describe}\PY{p}{(}\PY{p}{)}
\end{Verbatim}


\begin{Verbatim}[commandchars=\\\{\}]
{\color{outcolor}Out[{\color{outcolor}43}]:}             duration
         count  136734.000000
         mean        6.837652
         std         2.339572
         min         1.000000
         25\%         4.983333
         50\%         6.833333
         75\%         8.733333
         max        11.116667
\end{Verbatim}
            
    Agora o desvio padrão é um valor menor e os dados representam de forma
mais adequada os usuários do tipo 'subscriber'.

    \subsubsection{Para usuários do tipo
'customer':}\label{para-usuuxe1rios-do-tipo-customer}

A análise será feita no intervalo de 7 a 22h,visto que existe apenas um
intervalo de maior movimento e já tão acentuado como no caso do
'subscriber'. Intervalo definido com base no gráfico 'Busy Hours for
Customer'.

    \begin{Verbatim}[commandchars=\\\{\}]
{\color{incolor}In [{\color{incolor}44}]:} \PY{n}{customer\PYZus{}busy} \PY{o}{=} \PY{n}{customer}\PY{p}{[}\PY{p}{[}\PY{l+s+s1}{\PYZsq{}}\PY{l+s+s1}{duration}\PY{l+s+s1}{\PYZsq{}}\PY{p}{,} \PY{l+s+s1}{\PYZsq{}}\PY{l+s+s1}{weekday}\PY{l+s+s1}{\PYZsq{}}\PY{p}{,} \PY{l+s+s1}{\PYZsq{}}\PY{l+s+s1}{start\PYZus{}hour}\PY{l+s+s1}{\PYZsq{}}\PY{p}{]}\PY{p}{]}\PY{p}{[}\PY{p}{(}\PY{n}{customer}\PY{p}{[}\PY{l+s+s1}{\PYZsq{}}\PY{l+s+s1}{start\PYZus{}hour}\PY{l+s+s1}{\PYZsq{}}\PY{p}{]}\PY{o}{\PYZgt{}}\PY{o}{=}\PY{l+m+mi}{7}\PY{p}{)} \PY{o}{\PYZam{}} \PY{p}{(}\PY{n}{customer}\PY{p}{[}\PY{l+s+s1}{\PYZsq{}}\PY{l+s+s1}{start\PYZus{}hour}\PY{l+s+s1}{\PYZsq{}}\PY{p}{]}\PY{o}{\PYZlt{}}\PY{o}{=}\PY{l+m+mi}{22}\PY{p}{)}\PY{p}{]}
         
         \PY{c+c1}{\PYZsh{}testar funcionamento}
         \PY{k}{print} \PY{l+s+s1}{\PYZsq{}}\PY{l+s+s1}{The maximum start\PYZus{}hour is }\PY{l+s+s1}{\PYZsq{}}\PY{p}{,} \PY{n+nb}{max}\PY{p}{(}\PY{n}{customer\PYZus{}busy}\PY{p}{[}\PY{l+s+s1}{\PYZsq{}}\PY{l+s+s1}{start\PYZus{}hour}\PY{l+s+s1}{\PYZsq{}}\PY{p}{]}\PY{o}{.}\PY{n}{values}\PY{p}{)}
         \PY{k}{print} \PY{l+s+s1}{\PYZsq{}}\PY{l+s+s1}{The minimum start\PYZus{}hour is }\PY{l+s+s1}{\PYZsq{}}\PY{p}{,} \PY{n+nb}{min}\PY{p}{(}\PY{n}{customer\PYZus{}busy}\PY{p}{[}\PY{l+s+s1}{\PYZsq{}}\PY{l+s+s1}{start\PYZus{}hour}\PY{l+s+s1}{\PYZsq{}}\PY{p}{]}\PY{o}{.}\PY{n}{values}\PY{p}{)}
         \PY{k}{print} \PY{n+nb}{type}\PY{p}{(}\PY{n}{customer\PYZus{}busy}\PY{p}{)}
\end{Verbatim}


    \begin{Verbatim}[commandchars=\\\{\}]
The maximum start\_hour is  22
The minimum start\_hour is  7
<class 'pandas.core.frame.DataFrame'>

    \end{Verbatim}

    \begin{Verbatim}[commandchars=\\\{\}]
{\color{incolor}In [{\color{incolor}45}]:} \PY{c+c1}{\PYZsh{}Estatistica com describe() function}
         \PY{n}{customer\PYZus{}busy}\PY{p}{[}\PY{l+s+s1}{\PYZsq{}}\PY{l+s+s1}{duration}\PY{l+s+s1}{\PYZsq{}}\PY{p}{]}\PY{o}{.}\PY{n}{describe}\PY{p}{(}\PY{p}{)}
\end{Verbatim}


\begin{Verbatim}[commandchars=\\\{\}]
{\color{outcolor}Out[{\color{outcolor}45}]:} count     99353.000000
         mean         64.061153
         std         946.387841
         min           1.000000
         25\%          10.983333
         50\%          18.600000
         75\%          38.600000
         max      287840.000000
         Name: duration, dtype: float64
\end{Verbatim}
            
    \emph{Análise sobre os dados estatísticos:}

Os dados mostram um comportamento de trajetos mais longos, onde a média
é bem superior ao tempo sem taxamento extra (30min). Nesse caso, o
desvio padrão é é ainda mais alto, porém o terceiro quartil está mais
próximo da média, demonstrando que valores mais baixos estão causando um
empecilho para caracteriazação do comportamento do usuário do tipo
'customer'. Portanto, podemos limpar esses dados analisando apenas os
dados a partir do primeiro quartil (25\%) e analisar esse valor de
máximo que é de quase 200 dias.

    \begin{Verbatim}[commandchars=\\\{\}]
{\color{incolor}In [{\color{incolor}46}]:} \PY{c+c1}{\PYZsh{}Removendo dados a cima de 75\PYZpc{} dos dados}
         \PY{n}{customer\PYZus{}busy\PYZus{}clean} \PY{o}{=} \PY{n}{customer\PYZus{}busy}\PY{p}{[}\PY{p}{[}\PY{l+s+s1}{\PYZsq{}}\PY{l+s+s1}{duration}\PY{l+s+s1}{\PYZsq{}}\PY{p}{]}\PY{p}{]}\PY{p}{[}\PY{n}{customer\PYZus{}busy}\PY{p}{[}\PY{l+s+s1}{\PYZsq{}}\PY{l+s+s1}{duration}\PY{l+s+s1}{\PYZsq{}}\PY{p}{]} \PY{o}{\PYZgt{}}\PY{o}{=} \PY{n}{customer\PYZus{}busy}\PY{p}{[}\PY{l+s+s1}{\PYZsq{}}\PY{l+s+s1}{duration}\PY{l+s+s1}{\PYZsq{}}\PY{p}{]}\PY{o}{.}\PY{n}{quantile}\PY{p}{(}\PY{o}{.}\PY{l+m+mi}{25}\PY{p}{)}\PY{p}{]}
         \PY{n}{customer\PYZus{}busy\PYZus{}clean2} \PY{o}{=} \PY{n}{customer\PYZus{}busy\PYZus{}clean}\PY{p}{[}\PY{p}{[}\PY{l+s+s1}{\PYZsq{}}\PY{l+s+s1}{duration}\PY{l+s+s1}{\PYZsq{}}\PY{p}{]}\PY{p}{]}\PY{p}{[}\PY{n}{customer\PYZus{}busy\PYZus{}clean}\PY{p}{[}\PY{l+s+s1}{\PYZsq{}}\PY{l+s+s1}{duration}\PY{l+s+s1}{\PYZsq{}}\PY{p}{]} \PY{o}{\PYZlt{}}\PY{o}{=} \PY{l+m+mi}{100000}\PY{p}{]}
         
         
         \PY{c+c1}{\PYZsh{}customer\PYZus{}busy\PYZus{}clean2 = customer\PYZus{}busy\PYZus{}clean.sort\PYZus{}values(\PYZsq{}duration\PYZsq{}).drop([len(customer\PYZus{}busy\PYZus{}clean.sort\PYZus{}values(\PYZsq{}duration\PYZsq{}))\PYZhy{}1])}
         \PY{c+c1}{\PYZsh{}Testa para analisar se a limpeza funcionou}
         \PY{k}{print} \PY{n}{customer\PYZus{}busy\PYZus{}clean2}\PY{o}{.}\PY{n}{sort\PYZus{}values}\PY{p}{(}\PY{l+s+s1}{\PYZsq{}}\PY{l+s+s1}{duration}\PY{l+s+s1}{\PYZsq{}}\PY{p}{)}\PY{o}{.}\PY{n}{head}\PY{p}{(}\PY{p}{)}
         \PY{k}{print} \PY{n}{customer\PYZus{}busy\PYZus{}clean2}\PY{o}{.}\PY{n}{sort\PYZus{}values}\PY{p}{(}\PY{l+s+s1}{\PYZsq{}}\PY{l+s+s1}{duration}\PY{l+s+s1}{\PYZsq{}}\PY{p}{)}\PY{o}{.}\PY{n}{tail}\PY{p}{(}\PY{p}{)}
\end{Verbatim}


    \begin{Verbatim}[commandchars=\\\{\}]
         duration
340     10.983333
11485   10.983333
202113  10.983333
10064   10.983333
215348  10.983333
            duration
421839  11481.650000
606063  12007.566667
80510   12037.266667
371066  18892.333333
382718  35616.666667

    \end{Verbatim}

    \begin{Verbatim}[commandchars=\\\{\}]
{\color{incolor}In [{\color{incolor}47}]:} \PY{c+c1}{\PYZsh{}Reexecutando funcao describe() para os dados limpos}
         \PY{n}{customer\PYZus{}busy\PYZus{}clean2}\PY{o}{.}\PY{n}{describe}\PY{p}{(}\PY{p}{)}
\end{Verbatim}


\begin{Verbatim}[commandchars=\\\{\}]
{\color{outcolor}Out[{\color{outcolor}47}]:}            duration
         count  74540.000000
         mean      79.010043
         std      285.451038
         min       10.983333
         25\%       16.316667
         50\%       24.550000
         75\%       60.666667
         max    35616.666667
\end{Verbatim}
            
    Mesmo execuando essa limpeza, o desvio padrão apresentou um valor mais
alto, portanto, a média não é representativa para esses dados. Portanto,
sria necessário uma analise mais profunda sobre a representatividade
desses dados em relação a caracteriazação do comportamento dos usuários
do tipo 'customer'.

    \subsection{Análise dos dias da semana de maior utilização em relação
aos dois tipos de
usuários}\label{anuxe1lise-dos-dias-da-semana-de-maior-utilizauxe7uxe3o-em-relauxe7uxe3o-aos-dois-tipos-de-usuuxe1rios}

    Para essa análise o objetivo é plotar os dias da semana e número de
viagens realizadas para cada dia, comparando os dias de maior movimento
para os dois tipos de usuários.

    \subsubsection{Para usuários do tipo
'subscriber':}\label{para-usuuxe1rios-do-tipo-subscriber}

    \begin{Verbatim}[commandchars=\\\{\}]
{\color{incolor}In [{\color{incolor}48}]:} \PY{c+c1}{\PYZsh{}Filtrando para um Pandas.Series com apenas os weeksdays}
         \PY{n}{subscriber\PYZus{}weekday} \PY{o}{=} \PY{n}{subscriber}\PY{p}{[}\PY{l+s+s1}{\PYZsq{}}\PY{l+s+s1}{weekday}\PY{l+s+s1}{\PYZsq{}}\PY{p}{]}
         
         \PY{k}{print} \PY{n+nb}{type}\PY{p}{(}\PY{n}{subscriber\PYZus{}weekday}\PY{p}{)}
         \PY{k}{print} \PY{n}{subscriber\PYZus{}weekday}\PY{o}{.}\PY{n}{head}\PY{p}{(}\PY{p}{)}
         \PY{k}{print} \PY{n}{subscriber\PYZus{}weekday}\PY{o}{.}\PY{n}{tail}\PY{p}{(}\PY{p}{)}
\end{Verbatim}


    \begin{Verbatim}[commandchars=\\\{\}]
<class 'pandas.core.series.Series'>
0    3
1    3
2    3
3    3
4    3
Name: weekday, dtype: int64
669949    0
669951    0
669952    0
669953    0
669954    0
Name: weekday, dtype: int64

    \end{Verbatim}

    \begin{Verbatim}[commandchars=\\\{\}]
{\color{incolor}In [{\color{incolor}49}]:} \PY{c+c1}{\PYZsh{}Criando um histograma que mostre a frequencia de utilização para cada dia da semana}
         \PY{n}{subscriber\PYZus{}weekday}\PY{o}{.}\PY{n}{hist}\PY{p}{(}\PY{n+nb}{range}\PY{o}{=}\PY{p}{(}\PY{l+m+mi}{0}\PY{p}{,}\PY{l+m+mi}{7}\PY{p}{)}\PY{p}{,} \PY{n}{bins}\PY{o}{=}\PY{l+m+mi}{7}\PY{p}{)}
         \PY{n}{plt}\PY{o}{.}\PY{n}{xlim}\PY{p}{(}\PY{l+m+mi}{0}\PY{p}{,}\PY{l+m+mi}{7}\PY{p}{)}
         \PY{n}{plt}\PY{o}{.}\PY{n}{ylim}\PY{p}{(}\PY{l+m+mi}{0}\PY{p}{,}\PY{l+m+mi}{120000}\PY{p}{)}
         \PY{n}{plt}\PY{o}{.}\PY{n}{title}\PY{p}{(}\PY{l+s+s1}{\PYZsq{}}\PY{l+s+s1}{Weekday x Frequency \PYZhy{} Subscriber }\PY{l+s+s1}{\PYZsq{}}\PY{p}{)}
         \PY{n}{plt}\PY{o}{.}\PY{n}{xlabel}\PY{p}{(}\PY{l+s+s1}{\PYZsq{}}\PY{l+s+s1}{Weekdays}\PY{l+s+s1}{\PYZsq{}}\PY{p}{)}
         \PY{n}{plt}\PY{o}{.}\PY{n}{ylabel}\PY{p}{(}\PY{l+s+s1}{\PYZsq{}}\PY{l+s+s1}{Frequency}\PY{l+s+s1}{\PYZsq{}}\PY{p}{)}
\end{Verbatim}


\begin{Verbatim}[commandchars=\\\{\}]
{\color{outcolor}Out[{\color{outcolor}49}]:} <matplotlib.text.Text at 0xb5f9588>
\end{Verbatim}
            
    \begin{center}
    \adjustimage{max size={0.9\linewidth}{0.9\paperheight}}{output_107_1.png}
    \end{center}
    { \hspace*{\fill} \\}
    
    \emph{Análise do Gráfico 'Weekday x Frequency - Subscriber': }

Queda abrupta de utilização dos serviços aos finais de semana.

    \subsubsection{Para usuários do tipo
'customer':}\label{para-usuuxe1rios-do-tipo-customer}

    \begin{Verbatim}[commandchars=\\\{\}]
{\color{incolor}In [{\color{incolor}50}]:} \PY{c+c1}{\PYZsh{}Filtrando para um Pandas.Series com apenas os weeksdays}
         \PY{n}{customer\PYZus{}weekday} \PY{o}{=} \PY{n}{customer}\PY{p}{[}\PY{l+s+s1}{\PYZsq{}}\PY{l+s+s1}{weekday}\PY{l+s+s1}{\PYZsq{}}\PY{p}{]}
         
         \PY{k}{print} \PY{n+nb}{type}\PY{p}{(}\PY{n}{customer\PYZus{}weekday}\PY{p}{)}
         \PY{k}{print} \PY{n}{customer\PYZus{}weekday}\PY{o}{.}\PY{n}{head}\PY{p}{(}\PY{p}{)}
         \PY{k}{print} \PY{n}{customer\PYZus{}weekday}\PY{o}{.}\PY{n}{tail}\PY{p}{(}\PY{p}{)}
         \PY{k}{print} \PY{n+nb}{max}\PY{p}{(}\PY{n}{customer\PYZus{}weekday}\PY{o}{.}\PY{n}{values}\PY{p}{)}
\end{Verbatim}


    \begin{Verbatim}[commandchars=\\\{\}]
<class 'pandas.core.series.Series'>
24    3
30    3
41    3
48    3
51    3
Name: weekday, dtype: int64
669950    0
669955    0
669956    0
669957    0
669958    0
Name: weekday, dtype: int64
6

    \end{Verbatim}

    \begin{Verbatim}[commandchars=\\\{\}]
{\color{incolor}In [{\color{incolor}51}]:} \PY{c+c1}{\PYZsh{}Criando um histograma que mostre a frequencia de utilização para cada dia da semana}
         \PY{n}{customer\PYZus{}weekday}\PY{o}{.}\PY{n}{hist}\PY{p}{(}\PY{n+nb}{range}\PY{o}{=}\PY{p}{(}\PY{l+m+mi}{0}\PY{p}{,}\PY{l+m+mi}{7}\PY{p}{)}\PY{p}{,} \PY{n}{bins}\PY{o}{=}\PY{l+m+mi}{7}\PY{p}{)}
         \PY{n}{plt}\PY{o}{.}\PY{n}{xlim}\PY{p}{(}\PY{l+m+mi}{0}\PY{p}{,}\PY{l+m+mi}{7}\PY{p}{)}
         \PY{n}{plt}\PY{o}{.}\PY{n}{ylim}\PY{p}{(}\PY{l+m+mi}{0}\PY{p}{,}\PY{l+m+mi}{25000}\PY{p}{)}
         \PY{n}{plt}\PY{o}{.}\PY{n}{title}\PY{p}{(}\PY{l+s+s1}{\PYZsq{}}\PY{l+s+s1}{Weekday x Frequency \PYZhy{} Customer }\PY{l+s+s1}{\PYZsq{}}\PY{p}{)}
         \PY{n}{plt}\PY{o}{.}\PY{n}{xlabel}\PY{p}{(}\PY{l+s+s1}{\PYZsq{}}\PY{l+s+s1}{Weekdays}\PY{l+s+s1}{\PYZsq{}}\PY{p}{)}
         \PY{n}{plt}\PY{o}{.}\PY{n}{ylabel}\PY{p}{(}\PY{l+s+s1}{\PYZsq{}}\PY{l+s+s1}{Frequency}\PY{l+s+s1}{\PYZsq{}}\PY{p}{)}
\end{Verbatim}


\begin{Verbatim}[commandchars=\\\{\}]
{\color{outcolor}Out[{\color{outcolor}51}]:} <matplotlib.text.Text at 0xb5e7dd8>
\end{Verbatim}
            
    \begin{center}
    \adjustimage{max size={0.9\linewidth}{0.9\paperheight}}{output_111_1.png}
    \end{center}
    { \hspace*{\fill} \\}
    
    \emph{Análise do Gráfico 'Weekday x Frequency - Customer': }

Aumento de utilização do serviço aos finais de semana.

    \section{Pergunta 5a - Para
subscriber}\label{pergunta-5a---para-subscriber}

Explore os dados e faça um gráfico que demonstre alguma particularidade
dos dados:

Gráfico mostrando o número de usuários relacionados a cada hora do dia.

    \begin{Verbatim}[commandchars=\\\{\}]
{\color{incolor}In [{\color{incolor}52}]:} \PY{n}{subscriber\PYZus{}x} \PY{o}{=} \PY{n}{hours\PYZus{}subscriber}\PY{o}{.}\PY{n}{keys}\PY{p}{(}\PY{p}{)}
         \PY{n}{subscriber\PYZus{}y} \PY{o}{=} \PY{n}{hours\PYZus{}subscriber}\PY{o}{.}\PY{n}{values}\PY{p}{(}\PY{p}{)}
         
         \PY{n}{plt}\PY{o}{.}\PY{n}{plot}\PY{p}{(}\PY{n}{subscriber\PYZus{}x}\PY{p}{,} \PY{n}{subscriber\PYZus{}y}\PY{p}{,}  \PY{n}{label}\PY{o}{=}\PY{l+s+s2}{\PYZdq{}}\PY{l+s+s2}{Subscriber}\PY{l+s+s2}{\PYZsq{}}\PY{l+s+s2}{s}\PY{l+s+s2}{\PYZdq{}}\PY{p}{)}
         
         
         \PY{n}{customer\PYZus{}x} \PY{o}{=} \PY{n}{hours\PYZus{}customer}\PY{o}{.}\PY{n}{keys}\PY{p}{(}\PY{p}{)}
         \PY{n}{customer\PYZus{}y} \PY{o}{=} \PY{n}{hours\PYZus{}customer}\PY{o}{.}\PY{n}{values}\PY{p}{(}\PY{p}{)}
         
         \PY{n}{plt}\PY{o}{.}\PY{n}{plot}\PY{p}{(}\PY{n}{customer\PYZus{}x}\PY{p}{,} \PY{n}{customer\PYZus{}y}\PY{p}{,} \PY{n}{label}\PY{o}{=}\PY{l+s+s2}{\PYZdq{}}\PY{l+s+s2}{Customer}\PY{l+s+s2}{\PYZsq{}}\PY{l+s+s2}{s}\PY{l+s+s2}{\PYZdq{}}\PY{p}{)}
         
         
         \PY{n}{plt}\PY{o}{.}\PY{n}{xlabel}\PY{p}{(}\PY{l+s+s1}{\PYZsq{}}\PY{l+s+s1}{Start Hour}\PY{l+s+s1}{\PYZsq{}}\PY{p}{)}
         \PY{n}{plt}\PY{o}{.}\PY{n}{ylabel}\PY{p}{(}\PY{l+s+s1}{\PYZsq{}}\PY{l+s+s1}{Frequency}\PY{l+s+s1}{\PYZsq{}}\PY{p}{)}
         \PY{n}{plt}\PY{o}{.}\PY{n}{title}\PY{p}{(}\PY{l+s+s2}{\PYZdq{}}\PY{l+s+s2}{Busy Hours for Subscriber}\PY{l+s+s2}{\PYZsq{}}\PY{l+s+s2}{s and Customer}\PY{l+s+s2}{\PYZsq{}}\PY{l+s+s2}{s}\PY{l+s+s2}{\PYZdq{}}\PY{p}{)}
         \PY{n}{plt}\PY{o}{.}\PY{n}{axis}\PY{p}{(}\PY{p}{[}\PY{l+m+mi}{0}\PY{p}{,} \PY{l+m+mi}{23}\PY{p}{,} \PY{l+m+mi}{0}\PY{p}{,} \PY{l+m+mi}{83000}\PY{p}{]}\PY{p}{)}
         \PY{n}{plt}\PY{o}{.}\PY{n}{legend}\PY{p}{(}\PY{n}{loc}\PY{o}{=}\PY{l+s+s1}{\PYZsq{}}\PY{l+s+s1}{upper left}\PY{l+s+s1}{\PYZsq{}}\PY{p}{)}
         \PY{n}{plt}\PY{o}{.}\PY{n}{show}\PY{p}{(}\PY{p}{)}
\end{Verbatim}


    \begin{center}
    \adjustimage{max size={0.9\linewidth}{0.9\paperheight}}{output_114_0.png}
    \end{center}
    { \hspace*{\fill} \\}
    
    O que é interessante na visualização acima? Por que você a selecionou?

\textbf{Answer}: O interresante é ver os diferentes comportamento dos
usuários, onde os os usuários do tipo \emph{'subscriber'} apresentam
pico bem definidos nos horários de rotina de trabalho(8h, 12h, 18h)
caracterizando-o. Já os usuários do tipo \emph{'customer'} não
apresentam picos de movimento, apenas ocorre um pequeno aumento (em uma
taxa bem menor) entre o período das 9h as 18h, mostrando um
comportamento compatível com turistas.

Esse gráfico foi selecionado pelo fato de fornecer informações
significativas e marcantes sobre os tipos de usuários, podendo, assim,
caracterila-los em alguns aspectos.

    \section{Pergunta 5b}\label{pergunta-5b}

Faça um gráfico que demonstre alguma particularidade dos dados:

    Gráfico comparando a frequencia de utilização dos dois tipos de usuários
em todos os dias da semana.

    \begin{Verbatim}[commandchars=\\\{\}]
{\color{incolor}In [{\color{incolor}53}]:} \PY{c+c1}{\PYZsh{} Gráfico Final 2}
         \PY{n}{fig}\PY{p}{,} \PY{n}{ax} \PY{o}{=} \PY{n}{plt}\PY{o}{.}\PY{n}{subplots}\PY{p}{(}\PY{p}{)}
         \PY{n}{ax}\PY{o}{.}\PY{n}{hist}\PY{p}{(}\PY{p}{[}\PY{n}{customer\PYZus{}weekday}\PY{p}{,} \PY{n}{subscriber\PYZus{}weekday}\PY{p}{]}\PY{p}{,} \PY{n}{histtype}\PY{o}{=}\PY{l+s+s2}{\PYZdq{}}\PY{l+s+s2}{bar}\PY{l+s+s2}{\PYZdq{}}\PY{p}{,} \PY{n}{label}\PY{o}{=}\PY{p}{(}\PY{l+s+s2}{\PYZdq{}}\PY{l+s+s2}{Customer}\PY{l+s+s2}{\PYZdq{}}\PY{p}{,} \PY{l+s+s2}{\PYZdq{}}\PY{l+s+s2}{Subscriber}\PY{l+s+s2}{\PYZdq{}}\PY{p}{)}\PY{p}{,} \PY{n+nb}{range}\PY{o}{=}\PY{p}{(}\PY{l+m+mi}{0}\PY{p}{,}\PY{l+m+mi}{7}\PY{p}{)}\PY{p}{,} \PY{n}{bins}\PY{o}{=}\PY{l+m+mi}{7}\PY{p}{)}
         \PY{n}{ax}\PY{o}{.}\PY{n}{set\PYZus{}title}\PY{p}{(}\PY{l+s+s2}{\PYZdq{}}\PY{l+s+s2}{Weekday x Frequency \PYZhy{} Customer and Subscriber}\PY{l+s+s2}{\PYZdq{}}\PY{p}{)}
         \PY{n}{ax}\PY{o}{.}\PY{n}{set\PYZus{}xlabel}\PY{p}{(}\PY{l+s+s1}{\PYZsq{}}\PY{l+s+s1}{Weekday}\PY{l+s+s1}{\PYZsq{}}\PY{p}{)}
         \PY{n}{ax}\PY{o}{.}\PY{n}{set\PYZus{}ylabel}\PY{p}{(}\PY{l+s+s1}{\PYZsq{}}\PY{l+s+s1}{Frequency}\PY{l+s+s1}{\PYZsq{}}\PY{p}{)}
         \PY{n}{ax}\PY{o}{.}\PY{n}{set\PYZus{}xlim}\PY{p}{(}\PY{l+m+mi}{0}\PY{p}{,}\PY{l+m+mi}{7}\PY{p}{)}
         \PY{n}{ax}\PY{o}{.}\PY{n}{set\PYZus{}ylim}\PY{p}{(}\PY{l+m+mi}{0}\PY{p}{,}\PY{l+m+mi}{120000}\PY{p}{)}
         \PY{n}{ax}\PY{o}{.}\PY{n}{legend}\PY{p}{(}\PY{p}{)}
\end{Verbatim}


\begin{Verbatim}[commandchars=\\\{\}]
{\color{outcolor}Out[{\color{outcolor}53}]:} <matplotlib.legend.Legend at 0xd4b5400>
\end{Verbatim}
            
    \begin{center}
    \adjustimage{max size={0.9\linewidth}{0.9\paperheight}}{output_118_1.png}
    \end{center}
    { \hspace*{\fill} \\}
    
    O que é interessante na visualização acima? Por que você a selecionou?

\textbf{Answer}: O gráfico a cima mostra uma grande diferencça entre os
dois tipos de usuários emrelação aos dias da semana de utilização. Os
usuários do tipo \emph{'subscriber'} utilizam o serviço durando os dias
da semana e bem menos aos finais de semana. O oposto ocorre com os
usuários do tipo \emph{'customer'}. Os \emph{"subscriber's"} utilizam o
serviço como locomoção no dia a dia, em suas rotinas e não como lazer. O
oposto ocorre com os \emph{"customer's"} onde ocorre um aumento de
utilização do serviço aos finais de semana, porém a diferença de número
de usuários nos dias de semana não é tão diferente aos finais de semana
(quando compardo com os usuários do tipo \emph{'subscriber'}) portanto
esse tipo de usuários devem ser turistas, em sua maioria. A escolha
desse gráfico foi motivada pelo fato de representar um comportamento
significativo sobre os usuários, mostrando grandes diferenças nos tipos
de usuários.

    \subsection{Conclusões}\label{conclusuxf5es}

Parabéns pela conclusão do projeto! Esta é apenas uma amostragem do
processo de análise de dados: gerando perguntas, limpando e explorando
os dados. Normalmente, neste momento no processo de análise de dados,
você pode querer tirar conclusões realizando um teste estatístico ou
ajustando os dados a um modelo para fazer previsões. Há também muitas
análises potenciais que podem ser realizadas se evoluirmos o código
fornecido. Em vez de apenas olhar para o número de viagens no eixo de
resultados, você pode ver quais recursos afetam coisas como a duração da
viagem. Nós também não analisamos como os dados meteorológicos se
encaixam no uso de bicicletas.

    \section{Pergunta 6}\label{pergunta-6}

Pense em um tópico ou campo de interesse onde você gostaria de poder
aplicar as técnicas da ciência dos dados. O que você gostaria de
aprender com o assunto escolhido?

\textbf{Responda}: Eu gostaria de aplicar análise de dados para diversas
empresas e indústrias da minha cidade. Gostaria de aprender como
realizaar análise de dados para diferentes ramos de atividade,
contruindo, assim, um grande portifólio de projetos. Por outro lado, meu
objetivo é evoluir para o aprendizado de Machine Learning, visando me
tornar um especialista de Inteligência Artificial e conseguir, no
futuro, concretizar um sonho de fundar uma empresa de técnologia ou
solfware baseada nessas ferramentas.

    \begin{quote}
\textbf{Dica}: se quisermos compartilhar os resultados de nossa análise
com os outros, existe uma outra opção que não é enviar o arquivo jupyter
Notebook (.ipynb). Também podemos exportar a saída do Notebook de uma
forma que pode ser aberto mesmo para aqueles sem o Python instalado. No
menu \textbf{File} na parte superior esquerda, vá para o submenu
\textbf{Download as}. Você pode então escolher um formato diferente que
pode ser visto de forma mais geral, como HTML (.html) ou PDF (.pdf).
Você pode precisar de pacotes adicionais ou software para executar essas
exportações.
\end{quote}


    % Add a bibliography block to the postdoc
    
    
    
    \end{document}
